% ----------------------------------------------------------------------------------------------------------------------
% AI & Data Analytics Learning Cards - Released under the MIT License
% ----------------------------------------------------------------------------------------------------------------------


% User guides
% Latex: https://ftp.agdsn.de/pub/mirrors/latex/dante/macros/latex/base/usrguide.pdf
% Tabularray: https://mirror.physik.tu-berlin.de/pub/CTAN/macros/latex/contrib/tabularray/tabularray.pdf
% TColorbox: https://mirror.clientvps.com/CTAN/macros/latex/contrib/tcolorbox/tcolorbox.pdf
% Posterbox Tutorial: https://mirror.clientvps.com/CTAN/macros/latex/contrib/tcolorbox/tcolorbox-tutorial-poster.pdf#[0,{%22name%22:%22Fit%22}]
% Xcolor: https://mirror.clientvps.com/CTAN/macros/latex/contrib/xcolor/xcolor.pdf
% TikZ and PGFPlots: https://tikz.dev/ and https://tikz.dev/pgfplots/
%                    https://tug.ctan.org/info/visualtikz/VisualTikZ.pdf 
% SVG Package: https://ctan.org/pkg/svg?lang=en
% Geometry: https://ftp.gwdg.de/pub/ctan/macros/latex/contrib/geometry/geometry.pdf
% fontspec: https://texdoc.org/serve/fontspec/0
% fancyhdr: https://ftp.fau.de/ctan/macros/latex/contrib/fancyhdr/fancyhdr.pdf
% biblatex: https://distrib-coffee.ipsl.jussieu.fr/pub/mirrors/ctan/macros/latex/contrib/biblatex/doc/biblatex.pdf
% amsmath: https://www.latex-project.org/help/documentation/amsldoc.pdf
% grapheur: https://ctan.tetaneutral.net/graphics/pgf/contrib/tkz-grapheur/doc/tkz-grapheur-doc-en.pdf
% enumitem: https://mirror.ibcp.fr/pub/CTAN/macros/latex/contrib/enumitem/enumitem.pdf

\documentclass[8pt]{extreport}


% --- Hyphenation Rules --------------------------------------------------------
\usepackage[USenglish]{babel}

% --- Fonts --------------------------------------------------------------------

% Ensure proper font encoding
\usepackage[T1]{fontenc}

% Load system fonts via fontspec (requires XeLaTeX or LuaLaTeX)
% to get a list of installed fonts: luaotfload-tool --list=format 
\usepackage{fontspec}
\setmainfont{FreeSans}[Scale=1]
\setsansfont{FreeSans}[Scale=1]
\setmonofont{FreeMono}[Scale=1]

% font for the words "AI & Data Analytics" in the header
\newfontfamily\headerfont{QTFuturePoster} 


% Emoji support
\usepackage{emoji}

% --- Page layout --------------------------------------------------------------
\usepackage[
    % showframe, 
    nomarginpar, 
    a4paper,
    portrait, 
    right=0.75cm, 
    left=0.75cm, 
    top=1.25cm, 
    bottom=1cm,
    headsep=0.25cm,
    headheight=0.75cm,
    footskip=0.5cm,
]{geometry}


\usepackage{fancyhdr}
\pagestyle{fancy}


% --- TColorbox ---------------------------------------------------------------
\usepackage{tcolorbox}
\tcbuselibrary{skins,raster,listings,breakable,minted, poster}


% --- Tables -------------------------------------------------------------------
\usepackage{tabularray}

% --- Color Names -------------------------------------------------------------
\usepackage[dvipsnames]{xcolor}

% --- Graphics -----------------------------------------------------------------
\usepackage{graphicx}
\usepackage{svg}
\graphicspath{{./figures/}{./icons/}{./logos/}}
\usepackage{tikz}
\usepackage{pgfplots}
\pgfplotsset{compat=1.18}
\usetikzlibrary{shapes.geometric}    % for shapes
\usetikzlibrary{arrows.meta}         % for arrows, Arrow tip library 
\usetikzlibrary{positioning}         % for relative positioning
\usetikzlibrary{calc}                % for calculations to make complex coordinate calculations
%\usetikzlibrary{backgrounds}        % Background Library "defines background for pictures". To use this in a Tikzpicture, an option is passed, e.g. \begin{tikzpicture}[show background rectangle], with a background rectangle style defined before the picture. (e.g. \tikzset{background rectangle/.style={<define background rectangle style here>}}
% \usetikzlibrary{calendars}         % for calendar drawings
% \usetikzlibrary{fadings}           % for fading effects
% \usetikzlibrary{patterns}          % for patterns
% \usetikzlibrary{shadows}           % for shadow effects 
\usetikzlibrary{chains}            % for chain effects
\usetikzlibrary{fit}               % for fitting nodes together
% \usetikzlibrary{er}                % for entity relationship diagrams
% \usetikzlibrary{intersections}     % for calculating intersections of paths
% \usetikzlibrary{mindmap}           % for mind maps
\usetikzlibrary{matrix}            % for matrix drawings
\usetikzlibrary{angles}           % for angle markings
\usetikzlibrary{quotes}           % for quotes in angles
% \usetikzlibrary{trees}             % for tree drawings
% \usetikzlibrary{decorations.pathmorphing}  % for path decorations like zigzag lines
\usetikzlibrary{decorations.pathreplacing} % for braces and other path decorations
% \usetikzlibrary{decorations.markings}      % for markings along a path
% \usetikzlibrary{decorations.shapes}        % for shape decorations
% \usetikzlibrary{decorations.text}          % for text decorations along a path
% \usetikzlibrary{pgfplots.colorbrewer}      % for color maps

% \usepackage{forest}               % for drawing trees https://ctan.ceremade.dauphine.fr/graphics/pgf/contrib/forest/forest-doc.pdf

\usepackage{adjustbox}          % for adjusting boxes

\usepackage[labelfont=sf]{caption}  % for figure captions on non-floats

\newcommand{\flowchartstylesmaincolor}{OliveGreen!80!White}
\newcommand{\flowchartstyleslinewidth}{1.25pt}

\tikzstyle{diam} = [diamond, aspect=2, draw, fill=red!40, text width=6em,text centered, line width=\flowchartstyleslinewidth ]
\tikzstyle{block} = [rectangle, draw=\flowchartstylesmaincolor, text width=3cm, text centered, rounded corners, minimum height=2em, line width=\flowchartstyleslinewidth ]
\tikzstyle{round} = [circle, draw=\flowchartstylesmaincolor, text centered, rounded corners, minimum height=2em, line width=\flowchartstyleslinewidth ]
\tikzstyle{trap} = [trapezium, trapezium left angle=70, trapezium right angle=110, minimum height=1em, text centered, draw=\flowchartstylesmaincolor!50!red, fill=\flowchartstylesmaincolor!20!green!30, line width=\flowchartstyleslinewidth ]
\tikzstyle{rect} = [rectangle, minimum width=3cm, minimum height=1cm, text centered, draw=\flowchartstylesmaincolor, line width=\flowchartstyleslinewidth, align=center, inner sep=1em ]
\tikzstyle{line} = [draw=\flowchartstylesmaincolor, -latex, line width=\flowchartstyleslinewidth ]
\tikzstyle{circ} = [round, draw=\flowchartstylesmaincolor,minimum width=2cm, align=center, line width=\flowchartstyleslinewidth ]


  % load flowchart styles


% --- URL and href -----------------------------------------------------------
\colorlet{citecolor}{black}
\colorlet{linkcolor}{black}
\colorlet{urlcolor}{black}
\usepackage[
  bookmarks=true,
  breaklinks=true,
  pdfborder={0 0 0},
  citecolor=citecolor,
  linkcolor=linkcolor,
  urlcolor=urlcolor,
  colorlinks=true,
  linktocpage=false,
  hyperindex=true,
  colorlinks=true,
  linktocpage=false,
  linkbordercolor=white]{hyperref}

% --- Math Environments --------------------------------------------------------
\usepackage{amsmath}
\usepackage{nicematrix}
\NiceMatrixOptions{cell-space-limits = 1pt}
\usepackage{siunitx}

% --- Bibliography -------------------------------------------------------------
\usepackage[autostyle=true]{csquotes} % recommended before biblatex
\usepackage[backend=biber,style=numeric,sorting=nyt]{biblatex}
\addbibresource{../bibliography/references.bib}

\renewcommand{\bibfont}{\normalfont\footnotesize}


% --- Enumerations -----------------------------------------------------------------
\usepackage{enumitem}

\setlist{noitemsep}
\setlist[1]{labelindent=\parindent} % < Usually a good idea
\setlist[itemize]{leftmargin=*}
\setlist[itemize,1]{label=-}

% --- Helpers -----------------------------------------------------------------
\usepackage{lipsum}
\usepackage{multicol}


% --- Options ---------------------------------------------------------
% \pagestyle{empty}
% \setlength\parindent{0pt}
% \setlength{\tabcolsep}{2pt}
% \baselineskip=0pt
\setlength{\columnsep}{0.75cm}
% \setlength{\parskip}{0.1\baselineskip}

\setlist[enumerate]{itemsep=0.2em, topsep=0.2em, parsep=0em, partopsep=0em}

\captionsetup{font=small,labelfont={bf,sf}}
% \captionsetup[sub]{font=small,labelfont={bf,sf}}

\setcounter{secnumdepth}{4}


\newcommand{\maincolor}{LimeGreen!90!Gray}
\newcommand{\maincolspan}{4}
\newcommand{\graphscale}{0.9}

\title{AI and Data Analytics for AI Beginners}
\author{Bernhard Gerlach}
\date{\today}

% --- TColorbox Skins ----------------------------------------------------------

% Colorbox skin for bash listings
\tcbsubskin{mintedbash}{empty}{
        size=minimal,
        listing engine=minted,
        listing only,
        minted style=friendly,
        minted language=bash,
        minted options={
            fontsize=\footnotesize,
            breaklines,
            autogobble,
        },
        colback=gray!10!white,
        colframe=gray!10!white,
        listing only,
        left=0em,
        enhanced,
}

% Colorbox skin for python listings
\tcbsubskin{mintedpython}{empty}{
        size=minimal,
        listing engine=minted,
        listing only,
        minted style=friendly,
        minted language=python,
        minted options={
            fontsize=\footnotesize,
            breaklines,
            autogobble,
        },
        colback=gray!10!white,
        colframe=gray!10!white,
        listing only,
        left=0em,
        enhanced,
        before skip=0.3\baselineskip,
}

% Skin for header box
\tcbsubskin{headerboxskin}{empty}{
        size=minimal,
        coltitle=black!10!black,
}



% Skin for orange-based section boxes
\tcbsubskin{sectionboxskin}{standard}{
    size=title, 
    left=0.2em,
    right=0.2em,
    arc=0.25mm,
    colback=\maincolor!2!white,
    colframe=\maincolor!75!black,
    coltitle=\maincolor!10!black,
    colbacktitle=\maincolor!75!white,
    fonttitle=\sffamily\bfseries,
    toptitle=0mm,
    bottomtitle=0mm,
    fontupper=\sffamily\small,
    fontlower=\sffamily\small,
    lower separated=false,
    valign lower=center,
    halign=left,
    subtitle style={top=0.4mm, bottom=0mm, boxrule=0.2pt, boxsep=0.1mm,
        colback=\maincolor!50!\maincolor!20!white,
        colupper=\maincolor!50!gray,
        fontupper=\sffamily\bfseries\small,
        coltext=\maincolor!10!black,
        height=1.1\baselineskip,
      }, 
}


% Skin for sub-section boxes
\tcbsubskin{subsectionboxskin}{standard}{
    size=fbox, 
    top=0.4em,
    % left=0.2em,
    % right=0.2em,
    % arc=0.25mm,
    % top=0.4mm, 
    % bottom=0mm, 
    % boxrule=0.2pt, 
    % boxsep=0.1mm,
    frame empty,
    % colback=\maincolor!2!white,
    colback=white,
    colbacktitle=\maincolor!50!\maincolor!20!white,
    fonttitle=\sffamily\bfseries\small,
    coltitle=\maincolor!10!black,
    % height=1.1\baselineskip,    
}


% Skin for simple tables inside boxes
\tcbsubskin{boxtablesimple}{empty}{
    size=minimal,
    before skip=0.2\baselineskip,
    after skip=0.2\baselineskip,
    colback=white,
    colframe=gray,
    frame empty,
}


\tcbsubskin{sectionraster}{sectionboxskin}{
    raster columns=6,
    raster equal height=rows,
    raster row skip=0.25cm,
    raster column skip=0.25cm,
    skin=sectionboxskin,
}





% ----------------------------------------------------------------------------------------------------------------------
% Header and Footer Settings
% ----------------------------------------------------------------------------------------------------------------------

\fancyhead[L]{
    \LARGE {\headerfont AI \& Data Analytics} \textcolor{\maincolor}{for AI beginners}
}
\fancyhead[C]{}
\fancyhead[R]{}
\fancyfoot[L]{}
\fancyfoot[C]{}
\fancyfoot[R]{\thepage}


% ----------------------------------------------------------------------------------------------------------------------
% Main Document
% ----------------------------------------------------------------------------------------------------------------------


\begin{document}

\begin{multicols}{2}[
    ]

    \setcounter{tocdepth}{1} % Show sections
    %\setcounter{tocdepth}{2} % + subsections
    %\setcounter{tocdepth}{3} % + subsubsections
    %\setcounter{tocdepth}{4} % + paragraphs
    %\setcounter{tocdepth}{5} % + subparagraphs
    \tableofcontents

    % ==== Introduction =============================================================================
    \chapter{Introduction to AI and Data Analytics}
    \section{Industry 4.0}

    \begin{tcbitemize}[ skin=sectionraster, halign lower=center ]
        \tcbitem[title=Information Management Cycle, raster multicolumn=6]
        Information is the oil of the 21st century and analytics is the combustion engine.
        \tcblower
        \adjustbox{scale=\graphscale}{
            \begin{tikzpicture}[node distance=1cm and 1cm]

                % place nodes
                \coordinate (center) at (0,0) {};
                \node [block, above=of center] (information-user) {INFORMATION USER} ;
                \node [block, right=of center] (data-sources) {DATA SOURCES} ;
                \node [block, below=of center] (information-ressources) {INFORMATION RESSOURCES} ;
                \node [block, left=of center] (information-product) {INFORMATION PRODUCT} ;

                % draw edges
                \draw [line] (information-user) -| (data-sources) node[midway, above, xshift=5mm] {Requirement Analysis};
                \draw [line] (data-sources) |- (information-ressources) node[midway, below, xshift=5mm] {Integration};
                \draw [line] (information-ressources) -| (information-product) node[midway, below, xshift=-5mm] {Evaluation};
                \draw [line] (information-product) |- (information-user) node[midway, above, xshift=-5mm] {Deployment};

            \end{tikzpicture}
        }

        \tcbitem[title={Data Analytics, Big Data and AI Application Areas}, raster multicolumn=6]
        \begin{multicols}{2}
            \begin{itemize}
                \item Optimize Quality Assurance
                \item Predictive Maintenance
                \item Reduce Process Runtimes
                \item Batch-size 1 Production
                \item Decision Support
                \item Minimize Ramp-up
                \item Minimize Process Error
            \end{itemize}
        \end{multicols}

        \tcbitem[title=Data Analytics Decision Making, raster multicolumn=6, halign upper=center]
        \tcblower
        \adjustbox{scale=\graphscale}{
            \begin{tikzpicture}[node distance=0.5cm and 0.5cm]
                \matrix[
                    matrix of nodes,
                    nodes={ circ, text width=1cm, minimum width=1cm },
                    column sep=1cm
                ] (m1) {
                    |(big-data)| (BIG) DATA & |(smart-data)| SMART DATA & |(decisions)| DECISIONS \\
                };

                % draw arrows
                \draw [line] (big-data) -- (smart-data);
                \draw [line] (smart-data) -- (decisions);
            \end{tikzpicture}
        }

        \vspace{2ex}

        \adjustbox{scale=0.95}{
            \begin{tcbitemize}[raster columns=4, raster equal height=rows, raster column skip=0.25cm, size=small, top=2mm, bottom=2mm, arc=3mm, boxrule=0.75pt, colframe=\maincolor!80!White]
                \tcbitem
                detect, collect, determine, expand network
                \begin{itemize}
                    \item simulations
                    \item sensors
                    \item databases
                    \item documentations
                    \item applications
                \end{itemize}
                \tcbitem
                consolidate, reconstruct, model, verify, enrich, re-evaluate
                \begin{itemize}
                    \item extract, transform, load (ETL)
                    \item streaming
                    \item descriptive analysis
                    \item semantic integration
                    \item enrichment
                \end{itemize}
                \tcbitem
                sort, reorder, reproduce, reduce, compacting, validate
                \begin{itemize}
                    \item predictive analysis
                    \item prescriptive analysis
                    \item data mining
                    \item data exploration
                    \item machine learning
                \end{itemize}
                \tcbitem
                visualize, integrate, interpret, arrange, prioritize
                \begin{itemize}
                    \item visual analytics
                    \item KPI
                    \item cockpits and reports
                    \item mental models
                \end{itemize}
            \end{tcbitemize}
        }

    \end{tcbitemize}

    \section{History of AI}
    \subsection{Historical Developments}
    \subsection{AI Winter}
    \subsection{Notable Advances in AI}

    \section{Expert Systems}
    \subsection{Overview Over Expert Systems}
    \subsection{Introduction to Prolog}

    \section{Neuroscience}
    \subsection{The (Human) Brain}
    \subsection{Cognitive Processes}

    \section{Modern AI Systems}
    \subsection{Recent Developments in Hard- and Software}
    \subsection{Narrow vs General AI}
    \subsection{NLP and Computer Vision}

    \section{AI Application Areas}
    \subsection{Autonomous Vehicles \& Mobility}
    \subsection{Personalized Medicine}
    \subsection{FinTech}
    \subsection{Retail \& Industry}

    \section{Literature}
    \begin{itemize}
        \item Chowdhary, K. R. (2020). Fundamentals of Artificial Intelligence. Springer India.
        \item Russell, S., \& Norvig, P. (2022). Artificial intelligence. A modern approach (4th ed.). Pearson Education.
        \item Ward, J. (2020). The student's guide to cognitive neuroscience (4th ed.). Taylor \& Francis Group.
    \end{itemize}

    % ==== Mathematical Foundations =============================================================================
    \chapter{Mathematical Foundations}
    \section{Advanced Mathematics}
    \subsection{Calculus}
    \subsubsection{Differentiation}
    \begin{tcbitemize}[ skin=sectionraster ]
        \tcbitem[title=Gradient, raster multicolumn=6]
        The gradient for any function $y = f(x)$ is defined as follows. The gradient is also called the first derivative. The three notations $f'(x)$, $dy/dx$ and $df(x)/dx$ are interchangeable\textsuperscript{\cite[][p89]{yangMathematicsCivilEngineers2018} and \cite[][p393]{bronsteinTaschenbuchMathematik2001}}.
        \tcblower
        \begin{equation}
            f'(x) = \frac{dy}{dx} = \frac{df(x)}{dx} = \lim_{\Delta x \to 0} \frac{f(x + \Delta x) - f(x)}{\Delta x}
        \end{equation}

        \vspace*{4ex}
        \centering
        \captionsetup{hypcap=false}
        \begin{tikzpicture}
            \begin{axis}[
                    width=8cm,
                    height=6cm,
                    axis lines=middle,
                    axis line style={thick, -Stealth},
                    xlabel={$x$},
                    ylabel={$y = f(x)$},
                    xmin=0, xmax=3,
                    ymin=-0.5, ymax=5,
                    enlarge y limits={upper=0.05},
                    enlarge x limits={upper=0.05},
                    samples=100,
                    domain=0:2.2,
                    legend cell align=left,
                    legend style={font=\footnotesize, at={(1.1,1.1)}, anchor=north east, cells={anchor=west}},
                    clip=false,
                    ylabel style={at={(ticklabel cs:1.0)},anchor=south},
                    xlabel style={at={(1.0,0.1)},anchor=south},
                    % ytick={0,1,2,3,4},
                    ytick distance=1,
                    minor y tick num=0,
                    grid=both,
                    major grid style={draw=gray!20, thin},
                    minor grid style={draw=gray!10, very thin},
                ]
                % Plot y = x^2
                \addplot[thick, \maincolor!80!black, smooth] {x^2};
                \addlegendentry{$y = x^2$}

                % Tangent at point P (x=1, y=1)
                % Derivative: y' = 2x, at x=1: y'(1) = 2
                % Tangent line: y - 1 = 2(x - 1) => y = 2x - 1
                \addplot[thick, red, dashed, domain=0.5:2.5] {2*x - 1};
                \addlegendentry{Tangent $y = 2x - 1$ at $P$}

                % Mark point P
                \addplot[only marks, mark=*, mark size=2pt, blue] coordinates {(1,1)};
                \node[above left, font=\footnotesize] at (axis cs:1,1) {$P$};

                % Line M: vertical line from P down to x-axis (at x=1)
                \draw[thick, blue] (axis cs:1,1) -- (axis cs:1,0);
                \node[right, font=\footnotesize, blue] at (axis cs:1,0.5) {$y$};

                % Line N: horizontal line from P to x=2 (parallel to x-axis)
                \draw[thick, blue] (axis cs:1,1) -- (axis cs:2,1);

                % Vertical line at x=2 from x-axis up
                % y=x^2 at x=2: y=4, tangent y=2x-1 at x=2: y=3
                \draw[thick, blue] (axis cs:2,0) -- (axis cs:2,4);

                % Mark intersection points
                \addplot[only marks, mark=*, mark size=1.5pt, blue] coordinates {(2,1)};  % N meets vertical at x=2
                \addplot[only marks, mark=*, mark size=1.5pt, blue] coordinates {(2,3)};  % tangent at x=2
                \addplot[only marks, mark=*, mark size=1.5pt, blue] coordinates {(2,4)};  % curve at x=2

                % Label dy: from N (y=1) to tangent (y=3) at x=2
                \draw[thick, orange, decorate, decoration={brace, amplitude=4pt, mirror}] (axis cs:2.1,1) -- (axis cs:2.1,3);
                \node[right, font=\footnotesize, orange] at (axis cs:2.15,2) {$dy$};

                % Label delta y: from N (y=1) to curve (y=4) at x=2
                \draw[thick, purple, decorate, decoration={brace, amplitude=4pt, mirror}] (axis cs:2.4,1) -- (axis cs:2.4,4);
                \node[right, font=\footnotesize, purple] at (axis cs:2.45,2.5) {$\Delta y$};

                % Label x: from origin to where M hits x-axis (x=1)
                \draw[thick, gray, decorate, decoration={brace, amplitude=4pt, mirror}] (axis cs:0,-0.55) -- (axis cs:1,-0.55);
                \node[below, font=\footnotesize, gray] at (axis cs:0.5,-0.75) {$x$};

                % Label delta x = dx: from M (x=1) to x=2
                \draw[thick, gray, decorate, decoration={brace, amplitude=4pt, mirror}] (axis cs:1,-0.55) -- (axis cs:2,-0.55);
                \node[below, font=\footnotesize, gray] at (axis cs:1.5,-0.7) {$\Delta x = dx$};

                % Angle alpha where tangent crosses x-axis
                % Tangent y=2x-1 crosses x-axis at x=0.5
                % where A, B, and C are the vertices, and B is the apex
                \path (1,0) coordinate (A) -- (0.5,0) coordinate (B) -- (1,1) coordinate (C) pic ["$\alpha$", draw, -latex, angle radius=25pt] {angle};
                % \draw[thick, black] (axis cs:0.8,0) arc[start angle=0, end angle=63.43, radius=0.15cm];
                % \node[above right, font=\footnotesize] at (axis cs:0.7,0.05) {$\alpha$};

            \end{axis}
        \end{tikzpicture}
        \captionof{figure}{Graph of example function $f(x) = y = x^2$ with first derivative $f'(x) = 2x$, hence with tangent $y = mx + b = 2x - 1$ at point $P(1,1)$. The tangent's gradient is defined by angle $\alpha$, known as the angle of inclination of the tangent line.}\label{fig:derivative}

    \end{tcbitemize}
    \begin{enumerate}[label*=\arabic*.]
        \item ...
        \item Integration
        \item Partial Differentiation
        \item Vector Analysis
    \end{enumerate}

    \subsection{Integral Transformations}
    \begin{enumerate}[label*=\arabic*.]
        \item Convolution
        \item Complex Numbers
        \item Fourier Series
        \item Fourier Transformation
    \end{enumerate}

    \subsection{Vector Algebra}
    \subsubsection{Scalars and Vectors}

    \begin{tcbitemize}[ skin=sectionraster ]
        \tcbitem[title=N-Dimensional Vector, raster multicolumn=6]
        In broad terms, vectors are things one can add and linear functions are functions of vectors that respect vector addition\textsuperscript{\cite[][p.12]{cherneyLinearAlgebra2013}}. Order of components matters.

        \begin{equation}
            \vec{v} =
            \begin{pNiceMatrix}
                a_1 \\ a_2 \\ \vdots \\ a_n
            \end{pNiceMatrix}
            \neq
            \begin{pNiceMatrix}
                \CodeBefore
                \cellcolor[HTML]{FFFF88}{1-1,2-1}
                \Body
                a_2 \\ a_1 \\ \vdots \\ a_n
            \end{pNiceMatrix}
        \end{equation}

        \tcbitem[title=Length (Magnitude), raster multicolumn=6]
        Length (magnitude) of vector $\vec{x}$. Also called Euclidean norm or 2-norm \textsuperscript{\cite[][p.90]{cherneyLinearAlgebra2013}}.
        \tcblower
        \begin{equation}
            \hat{x} = \left\lvert \vec{x} \right\rvert = \sqrt{\sum_{i=1}^n x_i^2} = \sqrt{x_1^2 + x_2^2 + ... + x_n^2}
        \end{equation}

        \tcbitem[title=Normalized vector (unit vector), raster multicolumn=6]
        Normalized vector (unit vector) in the same direction as $\vec{x}$ \textsuperscript{\cite[][p.262]{cherneyLinearAlgebra2013}}.
        \tcblower
        \begin{equation}
            \frac{\vec{x}}{\left\lvert \vec{x} \right\rvert} =
            \begin{pNiceMatrix}
                \frac{x_1}{\left\lvert \vec{x} \right\rvert} \\ \frac{x_2}{\left\lvert \vec{x} \right\rvert} \\ \vdots \\ \frac{x_n}{\left\lvert \vec{x} \right\rvert}
            \end{pNiceMatrix}
        \end{equation}

    \end{tcbitemize}

    \subsubsection{Addition and Subtraction of Vectors}
    \subsubsection{Multiplication of Vectors, Vector Product, Scalar Product}
    \begin{tcbitemize}[ skin=sectionraster ]

        \tcbitem[title=Scalar Product (dot product), raster multicolumn=6]
        Scalar product (dot product) of vectors $\vec{v}$ and $\vec{u}$ \textsuperscript{\cite[][p.89]{cherneyLinearAlgebra2013}}. The dot-product of 2 orthogonal vectors is 0 \textsuperscript{\cite[][p.30]{fischerMaschinellesLernenFuer2024}}.
        \tcblower
        \begin{equation}
            \vec{v} \cdot \vec{u} =
            \begin{pNiceMatrix}
                v_1 \\ v_2 \\ \vdots \\ v_n
            \end{pNiceMatrix}
            \cdot
            \begin{pNiceMatrix}
                u_1 \\ u_2 \\ \vdots \\ u_n
            \end{pNiceMatrix} = \sum_{i=1}^n v_i u_i = v_1 u_1 + v_2 u_2 + ... + v_n u_n
        \end{equation}

        \tcbitem[title=Angle and Cosine Similarity, raster multicolumn=6]
        The angle $\Theta$ of vectors $\vec{v}$ and $\vec{u}$ \textsuperscript{\cite[][p.90]{cherneyLinearAlgebra2013}}.

        \begin{equation}
            \vec{u} \cdot \vec{v} = \left\lvert \vec{u} \right\rvert \left\lvert \vec{v} \right\rvert \cos \Theta
            \Rightarrow
            \cos \Theta = \frac{\vec{u} \cdot \vec{v}}{\left\lvert \vec{u} \right\rvert \left\lvert \vec{v} \right\rvert}
            \Rightarrow
            \Theta = \arccos \left( \frac{\vec{u} \cdot \vec{v}}{\left\lvert \vec{u} \right\rvert \left\lvert \vec{v} \right\rvert} \right)
        \end{equation}

        \vspace{2ex}

        Cosine similarity is $\cos \Theta$. In case of normalized vectors $\hat{u}$ and $\hat{v}$, the cosine similarity is their dot product $\cos \Theta = \hat{u} \cdot \hat{v}$. Two vectors are more similar
        the smaller the angle $\Theta$ between them. Hence, highest similarity $\rightarrow$ lowest similarity $\rightarrow$ highest similarity: $\cos \ang{0} = 1$ (pointing into same direction) $\rightarrow$ $\cos \ang{90} = 0$ (orthogonal vectors) $\rightarrow$ $\cos \ang{180} = -1$ (pointing into opposite directions) \textsuperscript{\cite[][p.31]{fischerMaschinellesLernenFuer2024}}.

    \end{tcbitemize}

    \subsection{Vector Calculus}
    \begin{enumerate}[label*=\arabic*.]
        \item Differentiation of Vectors
        \item Integration of Vectors
        \item Scalar and Vector Fields
        \item Vector Operators
    \end{enumerate}

    \subsection{Matrices and Vector Spaces}
    \begin{enumerate}[label*=\arabic*.]
        \item Basic Matrix Algebra and Systems of Linear Equations
        \item Transpose, Trace, Determinant, and Inverse of a Matrix
        \item Eigenvalues, Eigenvectors, and Diagonalization
        \item Tensors
    \end{enumerate}

    \subsection{Information Theory}
    \begin{enumerate}[label*=\arabic*.]
        \item Mean Squared Error (MSE) and Simple Linear Regression
        \item Area Under the ROC Curve and Gini Index
        \item Entropy
        \item Cross Entropy
    \end{enumerate}

    \section{Advanced Statistics}
    \subsection{Introduction to Statistics}
    \begin{enumerate}[label*=\arabic*.]
        \item Random Variables
        \item Kolmogorov Axioms
        \item Probability Distributions
        \item Decomposing probability distributions
        \item Expectation Values and Moments
        \item Central Limit Theorem
        \item Sufficient Statistics
        \item Problems of Dimensionality
        \item Component Analysis and Discriminants
    \end{enumerate}
    \subsection{Important Probability Distributions and their Applications}
    \begin{enumerate}[label*=\arabic*.]
        \item Binomial Distribution
        \item Gauss or Normal Distribution
        \item Poisson and Gamma-Poisson Distribution
        \item Weibull Distribution
    \end{enumerate}
    \subsection{Bayesian Statistics}
    \begin{enumerate}[label*=\arabic*.]
        \item Bayes’ Rule
        \item Estimating the Prior, Benford’s Law, Jeffry’s Rule
        \item Conjugate Prior
        \item Bayesian \& Frequentist Approach
    \end{enumerate}
    \subsection{Descriptive Statistics}
    \begin{enumerate}[label*=\arabic*.]
        \item Mean, Median, Mode, Quantiles
        \item Variance, Skewness, Kurtosis
    \end{enumerate}
    \subsection{Data Visualization}
    \begin{enumerate}[label*=\arabic*.]
        \item General Principles of Dataviz/Visual Communication
        \item 1D, 2D Histograms
        \item Box Plot, Violin Plot
        \item Scatter Plot, Scatter Plot Matrix, Profile Plot
        \item Bar Chart
    \end{enumerate}
    \subsection{Parameter Estimation}
    \begin{enumerate}[label*=\arabic*.]
        \item Maximum Likelihood
        \item Ordinary Least Squares
        \item Expectation Maximization (EM)
        \item Lasso and Ridge Regularization
        \item Propagation of Uncertainties
    \end{enumerate}
    \subsection{Hypothesis Test}
    \begin{enumerate}[label*=\arabic*.]
        \item Error of 1st and 2nd Kind
        \item Multiple Hypothesis Tests
        \item p-Value
    \end{enumerate}

    \section{Literature}
    \begin{itemize}
        \item Mathai, A. M., \& Haubold, H. J. (2017). Linear algebra, a course for physicists and engineers (1st ed.). De Gruyter.
        \item Riley, K. F., Hobson, M. P., \& Bence, S. J. (2006). Mathematical methods for physics and engineering (2nd ed.). Cambridge University Press.
        \item Yang, X.-S. (2018). Mathematics for Civil Engineers: An Introduction. Dunedin Academic Press.
        \item Bruce, P., \& Bruce, A. (2017). Statistics for data scientists: 50 essential concepts. O’Reilley Publishing.
        \item Downey, A. (2013). Think Bayes. O’Reilley Publishing.
        \item Downey, A. (2014). Think stats. O’Reilley Publishing.
        \item McKay, D. (2003). Information theory, inference and learning algorithms. Cambridge University Press.
        \item Reinhart, A. (2015). Statistics done wrong. No Starch Press.
    \end{itemize}

    % ==== Programming Foundations =============================================================================
    \chapter{Programming Foundations}
    \section{Programming with Python}
    \subsection{Introduction to Python}
    \subsubsection{Data structures}
    \lipsum[1][1]
    \subsubsection{Functions}
    \subsubsection{Flow control}
    \lipsum[1][1]
    \subsubsection{Input / Output}
    \subsubsection{Modules and packages}
    \lipsum[1][1]

    \subsection{Classes and Inheritance}
    \subsubsection{Scopes and namespaces}
    \lipsum[1][1]
    \subsubsection{Classes and inheritance}
    \subsubsection{Iterators and generators}
    \lipsum[1][1]

    \subsection{Errors and Exceptions}
    \subsubsection{Syntax errors}
    \subsubsection{Handling and raising exceptions}
    \lipsum[1][1]
    \subsubsection{User-defined exceptions}

    \subsection{Important Libraries}
    \subsubsection{Standard Python library}
    \subsubsection{Scientific calculations}
    \lipsum[1][1]
    \subsubsection{Speeding up Python}
    \subsubsection{Visualization}
    \lipsum[1][1]
    \subsubsection{Accessing databases}

    \subsection{Working with Python}
    \subsubsection{Virtual environments}
    \lipsum[1][1]
    \subsubsection{Managing packages}
    \subsubsection{Unit and integration testing}
    \lipsum[1][1]
    \subsubsection{Documenting code}

    \subsection{Version Control}
    \subsubsection{Introduction to version control}
    \subsubsection{Version control with GIT}
    \lipsum[1][1]

    \section{NumPy and Pandas}
    \subsection{Introduction to NumPy}
    \subsubsection{NumPy arrays}
    \lipsum[1][1]
    \subsubsection{Array indexing and slicing}
    \subsubsection{Array operations}
    \lipsum[1][1]
    \subsubsection{Broadcasting}

    \subsection{Introduction to Pandas}
    \subsubsection{Series and DataFrames}
    \lipsum[1][1]
    \subsubsection{Data indexing and selection}
    \subsubsection{Data cleaning and preparation}
    \lipsum[1][1]
    \subsubsection{Data aggregation and grouping}

    \subsection{Data visualization with Pandas}
    \subsubsection{Plotting with Pandas}
    \lipsum[1][1]
    \subsubsection{Customizing plots}

    \subsection{Working with time series data}
    \subsubsection{Date and time handling}
    \subsubsection{Resampling and frequency conversion}

    \subsection{Advanced Pandas techniques}
    \subsubsection{Merging and joining DataFrames}
    \subsubsection{Pivot tables and cross-tabulations}

    \subsection{Performance optimization}
    \subsubsection{Efficient data manipulation}
    \subsubsection{Memory management}

    \section{SciKit-Learn}
    \section{PyTorch}

    \section{Literature}
    \begin{itemize}
        \item Lutz, M. (2017). Learning Python (5th ed.). O'Reilly.
        \item Mathes, E. (2019). Python Crash Course (2nd ed.). No Starch Press.
    \end{itemize}

    % ==== Data Analytics Foundations =============================================================================
    \chapter{Data Analytics Foundations}
    \section{Data Preprocessing}
    \subsection{Data Cleaning}
    \lipsum[1][1-3]
    \subsection{Data Integration}
    \lipsum[2][1-3]
    \subsection{Data Transformation}
    \lipsum[3][1-3]
    \subsection{Data Reduction}
    \lipsum[4][1-3]

    \subsection{Foundations of Data Systems}
    \subsubsection{Reliability}
    \subsubsection{Scalability}
    \subsubsection{Maintainability}

    \subsection{Data Processing at Scale}
    \subsubsection{Batch Processing}
    \subsubsection{Stream and Complex Event Processing}

    \subsection{Microservices}
    \subsubsection{Introduction to Microservices}
    \subsubsection{Implementing Microservices}

    \subsection{Governance \& Security}
    \subsubsection{Data Protection}
    \subsubsection{Data Security}
    \subsubsection{Data Governance}

    \subsection{Common Cloud Platforms \& Services}
    \subsubsection{Amazon AWS}
    \subsubsection{Google Cloud}
    \subsubsection{Microsoft Azure}

    \subsection{Data Ops}
    \subsubsection{Defining Principles}
    \subsubsection{Containerization}
    \subsubsection{Building Data Pipelines}

    \section{Exploratory Data Analysis}
    \subsection{Descriptive Statistics}
    \lipsum[5][1-3]
    \subsection{Data Visualization Techniques}
    \lipsum[6][1-3]
    \subsection{Correlation Analysis}
    \lipsum[7][1-3]
    \subsection{Dimensionality Reduction}
    \lipsum[8][1-3]
    \section{Literature}
    \begin{itemize}
        \item Han, J., Kamber, M., \& Pei, J. (2011). Data mining: Concepts and techniques (3rd ed.). Morgan Kaufmann.
        \item Provost, F., \& Fawcett, T. (2013). Data science for business: What you need to know about data mining and data-analytic thinking. O'Reilly Media.
        \item Tukey, J. W. (1977). Exploratory data analysis. Addison-Wesley.
        \item Andrade, H., Gedik, B., \& Turaga, D. (2014). Fundamentals of stream processing: Application
              design, systems, and analytics. Cambridge University Press.
        \item Axelrod, C. W. (2013). Engineering safe and secure software systems. Artech House.
        \item Kleppmann, M. (2017). Designing data-intensive applications: The big ideas behind reliable,
              scalable, and maintainable systems. O'Reilly.
        \item Newman, S. (2015). Building microservices: Designing fine-grained systems. O'Reilly.
    \end{itemize}

    % ==== Machine Learning =============================================================================
    \chapter{Machine Learning}
    Test
    \section{Introduction to Machine Learning}
    Clustering and classification both end up with “groups” of data, but they differ fundamentally in what is given as input, what is learned, and which question is being answered.

    Supervised vs. Unsupervised
    Classification is supervised learning: the training data already comes with known class labels (for example “spam”/“not spam”, “dog”/“cat”), and the model learns a mapping from features to these predefined labels.
    Clustering is unsupervised learning: there are no labels; the algorithm is supposed to find “natural” groups in the data solely from the features, typically based on some notion of similarity or distance.

    Different objectives

    Classification answers the question: “To which of the known classes does this new object belong?” – it is a prediction task with a clearly defined output space.
    Clustering answers something like: “How can these data points be partitioned into meaningful groups if only their features are known but no classes?” – it focuses on discovering structure, segmentation, and often also on exploration or complexity reduction.

    Why separate algorithm families?

    Classification algorithms (logistic regression, SVM, random forest, k-NN in its usual form) optimize a loss function with respect to known labels, for example misclassification rate or cross-entropy.
    Clustering algorithms (k-means, hierarchical clustering, DBSCAN, Gaussian mixture models) typically optimize some measure of “within-cluster cohesion and between-cluster separation”, for example within-cluster variance or density criteria, without ever seeing any “right/wrong” label.

    \begin{tblr}{|c|X|X|}
        \hline
        Aspect             & Classification                                & Clustering                                           \\
        \hline
        Learning type      & Supervised (with labels)                      & Unsupervised (without labels)                        \\
        \hline
        Main question      & ``Which known class does x belong to?''       & ``What structure / groups are present in the data?'' \\
        \hline
        Output             & Predefined class labels                       & Newly discovered clusters                            \\
        \hline
        Typical algorithms & Logistic regression, SVM, random forest, k-NN & k-means, DBSCAN, hierarchical clustering, GMM        \\
        \hline
        Evaluation         & Accuracy, F1, ROC-AUC                         & Silhouette, within-cluster variance, domain judgment \\
        \hline
    \end{tblr}

    \subsection{Regression \& Classification}
    \lipsum[1][1-3]
    \subsection{Supervised \& Unsupervised Learning}
    \lipsum[2][1-3]
    \subsection{Reinforcement Learning}
    \section{Clustering}
    \subsection{Introduction to clustering}
    \lipsum[3][1-3]
    \subsection{K-Means}
    \lipsum[4][1-3]
    \subsection{Expectation Maximization}
    \subsection{DBScan}
    \subsection{Hierarchical Clustering}
    \section{Regression}
    \subsection{Linear \& Non-linear Regression}
    \subsection{Logistic Regression}
    \subsection{Quantile Regression}
    \subsection{Multivariate Regression}
    \subsection{Lasso \& Ridge Regression}
    \section{Support Vector Machines}
    \subsection{Introduction to Support Vector Machines}
    \subsection{SVM for Classification}
    \subsection{SVM for Regression}
    \section{Decision Trees}
    \subsection{Introduction to Decision Trees}
    \subsection{Decision Trees for Classification}
    \subsection{Decision Trees for Regression}
    \section{Genetic Algorithms}
    \subsection{Introduction to Genetic Algorithms}
    \subsection{Applications of Genetic Algorithms}
    \section{Literature}
    \begin{itemize}
        \item Akerkar, R., \& Sajja, P. S. (2016). Intelligent techniques for data science. Springer International Publishing.
        \item Hodeghatta, U. R., \& Nayak, U. (2017). Business analytics using R- A practical approach. Apress Publishing.
        \item Lahoz-Beltra, R. (2016). SGA: Simple Genetic Algorithm (SGA) in Python.
        \item Runkler, T. A. (2012). Data analytics: Models and algorithms for intelligent data analysis. Springer Vieweg Press.
        \item Skiena, S. S (2017). The data science design manual. Springer International Publishing. Database: Springer eBook Package English Computer Science.
    \end{itemize}

    % ==== Deep Learning =============================================================================
    \chapter{Deep Learning}
    \section{Introduction to Neural Network and Deep Learning}
    \subsection{The Biological Brain}
    \subsection{Perceptron and Multi-Layer Perceptrons}
    \section{Network Architectures}
    \subsection{Feed-Forward Networks}
    \lipsum[5][1-3]
    \subsection{Convolutional Networks}
    \lipsum[6][1-3]
    \subsection{Recurrent Networks, Memory Cells and LSTMs}
    \section{Neural Network Training}
    \subsection{Weight Initialization and Transfer Function}
    \lipsum[7][1-3]
    \subsection{Backpropagation and Gradient Descent}
    \lipsum[8][1-3]
    \subsection{Regularization and Overtraining}
    \section{Alternative Training Methods}
    \subsection{Attention}
    \subsection{Feedback Alignment}
    \subsection{Synthetic Gradients}
    \subsection{Decoupled Network Interfaces}
    \section{Further Network Architectures}
    \subsection{Generative Adversarial Networks}
    \subsection{Autoencoders}
    \subsection{Restricted Boltzmann Machines}
    \subsection{Capsule Networks}
    \subsection{Spiking Networks}
    \section{Literature}
    \begin{itemize}
        \item Chollet, F. (2017). Deep learning with Python. Shelter Island, NY: Manning.
        \item Efron, B., \& Hastie, T. (2016). Computer age statistical inference. Cambridge: Cambridge University Press.
        \item Geron, A. (2017). Hands-on machine learning with Scikit-Learn and TensorFlow. Boston, MA: O’Reilly Publishing.
        \item Goodfellow, I., Bengio, Y., \& Courville, A. (2016). Deep learning. Boston, MA: MIT Press.
        \item Russel, S., \& Norvig, P. (2010). Artificial intelligence – A modern approach (3rd ed.). Essex: Pearson.
    \end{itemize}

    % ==== Reinforcement Learning =============================================================================
    \chapter{Reinforcement Learning}
    \section{Introduction to Reinforcement Learning}
    \subsection{Understanding Reinforcement Learning}
    \lipsum[9][1-3]
    \subsection{Components of Reinforcement Learning Systems}
    \lipsum[10][1-3]
    \section{Markov Chains}
    \subsection{Markov Decision Process \& Markov Property}
    \lipsum[11][1-3]
    \subsection{Value Functions and Discounted Value Functions}
    \lipsum[12][1-3]
    \subsection{General Utility Function}
    \subsection{Actions \& Policy}
    \subsection{Bellman's Equation}
    \subsection{Value Iteration}
    \subsection{Markov Chain Monte Carlo (MCMC)}
    \section{Bandit}
    \subsection{Single-Arm Bandit}
    \subsection{Multi-Arm Bandit}
    \section{Q-Learning}
    \subsection{Time-difference Learning}
    \subsection{Reinforcement Learning with Neural Networks \& Deep Q Learning}
    \subsection{Experience Replay}
    \subsection{Double Q-Learning}
    \subsection{Delayed Sparse Rewards}
    \subsection{Hierarchical Learning}
    \subsection{Value- vs Policy-Based Learning}
    \subsection{Actor-Critic Learning}
    \section{Reinforcement Learning Approaches}
    \subsection{Model-Free Learning}
    \subsection{Model-Based Learning}
    \subsection{Exploration vs Exploitation}

    \section{Inference and Causality}

    \subsection{Statistical Inference}
    \subsubsection{Bayesian inference}
    \lipsum[13][1-3]
    \subsubsection{Bayesian networks}
    \lipsum[14][1-3]
    \subsubsection{Probabilistic modelling}
    \lipsum[1][1]
    \subsection{Introduction to Causality}
    \subsubsection{Correlation vs causation}
    \lipsum[15][1-3]
    \subsubsection{Granger causality}
    \lipsum[16][1-3]
    \subsubsection{Directed Acyclic Graphs (DAG)}
    \lipsum[1][1]
    \subsubsection{Elements of causal graphs: collider, chain, fork}
    \lipsum[1][1]
    \subsubsection{D-separation}

    \subsection{Interventions}
    \subsubsection{Seeing vs doing}
    \lipsum[1][1]
    \subsubsection{Conditional independence}
    \lipsum[1][1]
    \subsubsection{Confounders \& counterfactuals}
    \lipsum[1][1]
    \subsubsection{Causal inference vs randomized controlled trials}
    \lipsum[1][1]

    \subsection{Do-calculus}
    \subsubsection{Front- \& backdoor criterion}
    \lipsum[1][1]
    \subsubsection{Three rules of do-calculus}
    \lipsum[1][1]

    \subsection{Fallacies}
    \subsubsection{Mediation fallacy}
    \lipsum[1][1]
    \subsubsection{Collider bias}
    \lipsum[1][1]
    \subsubsection{Simpson’s \& Berkson’s Paradox}
    \lipsum[1][1]
    \subsubsection{Imputing missing values: causal vs data-driven view}
    \lipsum[1][1]

    \section{Literature}
    \begin{itemize}
        \item Berzuini, C., Dawid, P., \& Bernardinelli, L. (2012). Causality: Statistical perspectives and applications. Wiley.
        \item Hernan, M. A., \& Robins, J. M. (2020). Causal inference: What if. CRC Press.
        \item Pearl, J. (2013). Causality: Models, reasoning and inference (2nd ed.). Cambridge University Press.
        \item Pearl, J., \& Mackenzie, D. (2018). The book of why: The new science of cause and effect. Basic Books.
        \item Pearl, J., Glymour, M., \& Jewell, N. P. (2016). Causal inference in statistics: A primer. Wiley.
        \item Wakefield, J. (2013). Bayesian and frequentist regression methods. Springer.
        \item Bertsekas, D. P. (2019). Reinforcement learning and optimal control. Athena Scientific.
        \item Sutton, R. S., \& Barto, A. G. (1998). Reinforcement learning: An introduction. MIT Press.

    \end{itemize}

    % ==== Functional Security in AI Systems =============================================================================
    \chapter{Functional Security in AI Systems}
    \section{Introduction}
    \subsection{Functional Security}
    \subsection{Automotive Safety}
    \subsection{Relevant Standards (ISO 26262, IEC 61508, ISO 27001, EU directive 2001/95/EG, ISO 25119)}
    \section{Functional Safety Standard ISO 26262}
    \subsection{Introduction}
    \subsection{Automotive Safe Integrity Levels (ASIL)}
    \subsection{Recommended Techniques}
    \section{IT Security Standards}
    \subsection{ISO 27001}
    \subsection{ISO 15408}
    \subsection{ISO 21434}
    \subsection{SAE J3061}
    \subsection{AECQ}
    \section{Approaches}
    \subsection{Safe Failure Fraction (SFF)}
    \subsection{Diagnostic Coverage (DC)}
    \subsection{Hazard Analysis and Risk Assessment (HARA)}
    \subsection{Fault Tree Analysis (FTA)}
    \subsection{Failure Modes, Effects \& (Diagnostic, Criticality) Analysis (FME[C,D]A)}
    \section{Attacks and Defenses}
    \subsection{Cyber Attacks}
    \subsection{Physical Attacks}
    \subsection{MISRA C/C++ Guidelines}

    \section{Literature}
    \begin{itemize}
        \item Smith D.J., Simpson K. (2016). The Safety Critical Systems Handbook. (4th ed.) Elsevier
        \item Smith D.j. (2017). Reliability, Maintainability and Risk (9th ed.) Elsevier
        \item Rausand, M. (2014). Reliability of Safety-Critical Systems: Theory and Applications Wiley.
    \end{itemize}

    % ==== Foundational Computer Vision =============================================================================
    \chapter{Foundational Computer Vision}
    \section{Image Processing and Low Level Vision}

    \subsection{Image Acquisition}
    \subsubsection{The Human Visual System}
    \lipsum[17][1-3]
    \subsubsection{Cameras and Sensors}
    \lipsum[18][1-3]

    \subsection{Single and Multi-View Geometry}
    \subsubsection{Camera Geometry and Perspective Projection}
    \lipsum[1][1]
    \subsection{Stereopsis and Multiple Views}

    \subsection{Image Representation and Morphology}
    \subsubsection{Image Types}
    \lipsum[1][1]
    \subsubsection{Morphology of Binary and Greyscale Images}
    \lipsum[1][1]

    \subsection{Filtering}
    \subsubsection{Filtering in the Spatial Domain}
    \lipsum[1][1]
    \subsubsection{Fourier Transformation and Filtering in the Frequency Domain}
    \lipsum[1][1]

    \subsection{Texture}
    \subsubsection{Classical Texture Representations}
    \lipsum[1][1]
    \subsubsection{Bag of Words and Representation in CNNs}
    \lipsum[1][1]

    \section{Mid-Level Vision and Video}
    \subsection{Mid-Level Image Features}
    \subsubsection{Edges \& Lines}
    \lipsum[19][1-3]
    \subsubsection{Corners, Points of Interest, and Blobs}
    \lipsum[20][1-3]
    \subsubsection{Feature Based Alignment}
    \lipsum[1][1]

    \subsection{Segmentation}
    \subsubsection{Region Based Segmentation}
    \lipsum[1][1]
    \subsubsection{Contour Based Segmentation}
    \lipsum[1][1]

    \subsection{Motion}
    \subsubsection{Optical Flow}
    \lipsum[1][1]
    \subsubsection{Classical Approaches}
    \lipsum[1][1]
    \subsubsection{CNN Based Methods}

    \subsection{Tracking}
    \subsubsection{Kalman Filters}
    \subsubsection{Particle Filters}
    \subsubsection{Tracking Via Deep Networks}

    \subsection{Shape}
    \subsubsection{Shape from X}
    \subsubsection{Geometric Methods}
    \subsubsection{Radiometric Approaches}

    \section{Computer Vision for Autonomous Systems}
    \subsection{Image Formation \& Acquisition}
    \subsubsection{Light}
    \subsubsection{Color}
    \subsubsection{Perspective Camera}
    \subsubsection{Camera Calibration}
    \subsubsection{Single and Multiple View Geometry}
    \subsection{Sensors for Computer Vision}
    \subsubsection{Camera \& Night Vision}
    \lipsum[1][1]
    \subsubsection{Lidar}
    \subsubsection{Radar}
    \subsubsection{Ultrasound}
    \subsubsection{Trends}
    \subsection{Image Processing}
    \subsubsection{Operators}
    \subsubsection{Filtering and Transforms}
    \subsubsection{Geometric Transformations}
    \subsection{Feature Detection}
    \subsubsection{Points}
    \subsubsection{Edges}
    \subsubsection{Lines}
    \subsubsection{Common Methods}
    \subsection{Object Detection \& Tracking}
    \subsubsection{Object Representation}
    \subsubsection{Techniques for Object Detection}
    \subsubsection{Network Architectures}
    \subsection{Segmentation}
    \subsubsection{Stuff and Things}
    \subsubsection{Semantic Segmentation}
    \subsubsection{Instance Segmentation}
    \subsubsection{Segmentation in Videos and Feeds}
    \subsubsection{MOTS: Multi-Object Tracking \& Segmentation}

    \section{Cognitive Computer Vision}
    \subsection{Classification}
    \subsubsection{Classifying Image Content}
    \subsubsection{Image Retrieval and Tagging}
    \subsubsection{Scene Understanding}
    \subsection{Recognition}
    \subsubsection{Object Recognition}
    \subsubsection{Semantic and Instance Segmentation}
    \subsubsection{Face Recognition}
    \subsection{Image Synthesis}
    \subsubsection{Extrapolation}
    \subsubsection{Synthesizing Images and Image Enhancement (Superresolution)}
    \subsubsection{Artistic Style Transfer / Domain Transfer}
    \subsection{Computer Vision and NLP}
    \subsubsection{Scene Description}
    \subsubsection{Visual Question Answering}
    \subsubsection{Synthesizing Images from Descriptions}
    \subsection{Current Challenges}
    \subsubsection{Fairness and Explainability in Computer Vision}
    \subsubsection{Self Supervised and Contrastive Learning}
    \subsubsection{Robust Computer Vision}

    \section{Advanced NLP and Computer Vision}
    \subsection{Text Processing}
    \subsubsection{Machine translation}
    \subsubsection{Information extraction}
    \subsection{Speech Signal Processing}
    \subsubsection{Speech recognition}
    \subsubsection{Speech synthesis}
    \subsection{Geometry Reconstruction}
    \subsubsection{3D reconstruction from 2D images/videos}
    \subsubsection{Change of perspective}
    \subsection{Semantic Image Analysis}
    \subsubsection{Image retrieval}
    \subsubsection{Semantic segmentation / object detection}
    \subsubsection{Medical imaging analysis}
    \subsubsection{Copyright violation, counterfeit and forgery detection}
    \subsubsection{Face recognition and biometrics}
    \subsection{Tracking}
    \subsubsection{Challenges in tracking}
    \subsubsection{Object representation}
    \subsubsection{Single vs. multiple object tracking}

    \section{Literature}
    \begin{itemize}
        \item Forsyth, D., Ponce, J. (2012): Computer Vision - A Modern Approach, Prentice Hall.
        \item Gonzalez, R.C., Woods, R.E. (2017): Digital Image Processing (4th edition), Prentice-Hall.
        \item Hartley, R., Zisserman, A. (2004): Multiple View Geometry in Computer Vision, 2nd Edition, Cambridge University Press.
        \item Klette, R. (2014): Concise Computer Vision: An Introduction into Theory and Algorithms, Springer.

        \item Davies, E.R. (2012). Computer and Machine Vision. 4th edition. Academic Press. London, Oxford, Boston, New York and San Diego.
        \item Szeliski, R. (2010): Computer Vision - Algorithms and Applications, Springer.
        \item Ansari, S. (2020). Building Computer Vision Applications Using Artificial Neural Networks. Apress. https://doi.org/10.1007/978-1-4842-5887-3
        \item Ayyadevara, V., \& Reddy, Y. (2020). Modern Computer Vision with PyTorch. Packt.
        \item Distante, A., \& Distante, C. (2020). Handbook of image processing and computer vision: Volume 1: From energy to image. Springer International Publishing. https://doi.org/10.1007/978-3-030-38148-6
        \item Peters, J. F. (2017). Foundations of Computer Vision (Vol. 124). Cham: Springer International Publishing. https://doi.org/10.1007/978-3-319-52483-2
        \item Szelinski, R. (2020). Computer Vision: Algorithms and Applications. (2nd ed.). Springer Nature.
        \item Goodfellow, I., Bengio, Y., \& Courville, A. (2016). Deep Learning. MIT Press. Cambridge, MA.
        \item Dadhich, A. (2018). Practical Computer Vision. Pakt Publishing. Burmingham, UK.
        \item Hornberg, A. (2017). Handbook of Machine and Computer Vision. 2nd edition. Wiley-VCH. Weinheim, Germany.
        \item Shanmugamani, R. (2018). Deep Learning for Computer Vision. Pakt Publishing. Burmingham, UK.
        \item Solem, J. E. (2012). Programming Computer Vision with Python. O'Reilly. Cambridge, MA.
        \item Bengfort, B., Bilbro, R., \& Ojeda, T. (2018). Applied text analysis with Python: Enabling language-aware data products with machine learning. O'Reilly.
        \item Clark, A., Fox, C., \& Lappin, S. (Eds.). (2010). The handbook of computational linguistics and natural language processing. Wiley-Blackwell.
        \item Davies, E. R. (2017). Computer vision: Principles, algorithms, applications, learning (5th ed.). Academic Press.
        \item Fisher, R. B., Breckon, T. P., Dawson-Howe, K., Fitzgibbon, A., Robertson, C., Trucco, E., \& Williams, C. K. I. (2016). Dictionary of computer vision and image processing (2nd ed.). Wiley.

    \end{itemize}

    % ==== Artificial Intelligence in FinTech =============================================================================
    \chapter{Artificial Intelligence in FinTech}
    \section{Concepts of FinTechs and Artificial Intelligence}
    \subsection{Introduction of FinTechs and AI}
    \subsubsection{Definition of FinTechs and AI}
    \lipsum[21][1-3]
    \subsubsection{FinTech Ecosystem}
    \lipsum[22][1-3]
    \subsubsection{Revolution in the Financial Services Industry}
    \lipsum[1][1]
    \subsubsection{Open Banking Regulation}

    \subsection{Application of FinTechs in Banking and Finance}
    \subsubsection{Retail Banking}
    \subsubsection{Payment Transactions}
    \subsubsection{Wealth Management}
    \subsubsection{Financing}
    \lipsum[1][1]
    \subsubsection{Scope of FinTech in Financial Inclusion}
    \lipsum[1][1]

    \subsection{FinTech and AI Underlying Technologies}
    \subsubsection{Contemporary Developments in Banking Technology}
    \subsubsection{Cloud Banking}
    \subsubsection{Blockchain, DLT and Smart Contracts}
    \subsubsection{Machine and Deep Learning}
    \subsubsection{Neuroscience in Finance}
    \lipsum[1][1]

    \subsection{AI Application in the Financial Services Industry}
    \subsubsection{AI in Deposits and Lending}
    \subsubsection{Chatbots in Banking}
    \subsubsection{AI Use in Developing Credit Scoring Models}
    \subsubsection{AI in the Insurance Sector}
    \subsubsection{KYC and AML}

    \subsection{Trust and Ethical Issues Related to AI and FinTech}
    \subsubsection{Bias and Algorithmic Discrimination}
    \subsubsection{GDPR Directive in Europe}
    \subsubsection{Contemporary Regulatory Landscapes in Other Jurisdictions}

    \subsection{Future of FinTech and AI}
    \subsubsection{Building Trust from Past Events}
    \subsubsection{New Collaboration Opportunities}
    \subsubsection{Future of Banking Technology}
    \subsubsection{Role of FinTech and AI Start-ups in Sustainable and ESG Financing}
    \subsubsection{Future of Banking, Cryptocurrencies and CBDCs}
    test test
    \lipsum[1][1]

    \section{Fraud Detection FinTechs}
    \subsection{Introduction of Fraud Detections FinTechs}
    \subsubsection{The Exponential Growth of FinTechs}
    \subsubsection{The Importance of Fraud Detection and Prevention in FinTechs}
    \subsubsection{Wirecard FinTech Fraud in Germany}
    \subsubsection{Examples of FinTech Companies Detecting Fraud}

    \subsection{Insurance Fraud}
    \subsubsection{Nature of Insurance Frauds}
    \subsubsection{Application of Advanced Analytics for Fraud Detection}
    \subsubsection{Case Studies such as the OneConnect Smart Insurance Platform}

    \subsection{Money Laundering}
    \subsubsection{Overview of Cross-board Payments}
    \subsubsection{AI Use in Crypto-assets Fraud Detection}
    \subsubsection{Regtech and Machine Learning for Fraud Detection}
    \subsubsection{Regulatory Fines and Case Studies (HSBC, BNP Paribas)}

    \subsection{Identity Fraud}
    \subsubsection{Fraud of Personal Data}
    \subsubsection{Fraud Detection in Account Opening Process}
    \subsubsection{Accounts and Transaction Frauds}

    \subsection{Key Application Areas of AI Anomaly Detection in Financial Institutions}
    \subsubsection{Lending}
    \subsubsection{Asset Management}
    \subsubsection{Payments}
    \subsubsection{AI and Predictive Analytics}

    \subsection{Key Challenges of AI Use in Fraud Detection in Financial Institutions}
    \subsubsection{Quality of Data}
    \subsubsection{Lack of Qualified Staff}
    \subsubsection{Regulatory Issues}
    \subsubsection{Implementation of Technology such as Biometrics}
    \subsubsection{Regulatory Fines and Case Studies (HSBC, BNP Paribas)}

    \section{Robo Advisory}
    \subsection{Introduction of Robo Advisory}
    \subsubsection{Definition of Robo advisors}
    \subsubsection{Importance of Wealth and Asset Management Sector}
    \subsubsection{Issues of Traditional Asset Management Sector}
    \subsubsection{Drivers, History and Rise of Robo-advisory and Current State of Financial Markets}
    \subsection{Types of Robo-Advisors and their Business Models}
    \subsubsection{Machine Learning Types and Use in Finance}
    \subsubsection{Hybrid/Bionic Robo Advisors}
    \subsubsection{Pure Robo Advisors}
    \subsubsection{Issues of Supervised vs. Unsupervised}
    \subsubsection{Case studies such as Charles Schwab, Vanguard, Wealthfront and Betterment}
    \subsection{Selected Areas of Application}
    \subsubsection{Non-life Insurance}
    \subsubsection{Chatbots}
    \subsubsection{Stock and Derivative Price Predictions}
    \subsubsection{Automatic Rebalancing of Portfolios}
    \subsubsection{Tax-loss Harvesting}
    \subsection{Principles of Goal-Based Investing}
    \subsubsection{Investors Preferences}
    \subsubsection{Goal-based Investing Process}
    \subsubsection{Portfolio Modeling}
    \subsubsection{Risk Tolerance Framework}
    \subsection{Robo Economicus and Quantitative Models}
    \subsubsection{Home Bias}
    \subsubsection{Behavioral Accounting}
    \subsubsection{Quantitative Approaches in Robo-Advisory Models}
    \subsubsection{Risk Targets}
    \subsection{The Future Role of Robo-advisors and Asset Management Industry}
    \subsubsection{Future of Digital Advice}
    \subsubsection{Machine Learning Models in Python}
    \subsubsection{Frauds in Robo Advisory}
    \subsubsection{The Regulatory Landscape of Robo-Advisory in Germany and Europe}

    \section{Literature}
    \begin{itemize}
        \item Arjundwadkar, P.Y. (2018). FinTech: The technology driving disruption in the financial services industry. CRC Press (Taylor \& Francis Group), London.
        \item Chishti, S., Bartoletti, I., Leslie, A. \& Millie, S.M. (2020). The AI Book: The Artificial Intelligence handbook for investors, entrepreneurs and FinTech visionaries. Wiley, West Sussex.
        \item Lui, A. \& Ryder, N. (2021). FinTech, Artificial Intelligence and the Law: Regulation and Crime Prevention. Routledge, London.
        \item Arslanian, H. \& Fischer, F. (2019). The Future of Finance: The impact of FinTech, AI, and crypto on financial services. Palgrave Macmillan, Cham.
        \item Ashfaq, M. \& Randall, V.J. (2020). Wirecard: The Rise and Fall of a German FinTech. The Case Centre, UK.
        \item Boobier, T. (2020). AI and the Future of Banking. John Wiley \& Sons Ltd, West Sussex.
        \item Gough, L. (2021). The CON Men: A History of Financial Fraud and the Lessons You Can Learn. Pearson, Harlow.
        \item Narang, R. (2009). Inside the black box: the simple truth about quantitative trading. John Wiley \& Sons Ltd, New Jersey.
        \item Peter, S. (2021). Robo-Advisory: investing in the digital age. Palgrave Macmillan, Cham.
        \item Sironi, P. (2016). FinTech innovation: from Robo-advisors to goal based investing and gamification. John Wiley \& Sons Ltd, West Sussex.
        \item Tatsat, H., Puri, S., \& Lookbaugh, B. (2021). Machine learning and data science blueprints for finance: from building trading strategies to Robo-advisors using Python. O'Reilly, Boston.
    \end{itemize}

    % ==== AI in Healthcare and Medical Imaging =============================================================================
    \chapter{AI in Healthcare and Medical Imaging}
    \section{AI in Healthcare}
    \subsection{Healthcare Stakeholders}
    \subsubsection{Healthcare Management}
    \lipsum[23][1-3]
    \subsubsection{Insurance \& Intermediaries}
    \lipsum[24][1-3]
    \subsubsection{Pre-Clinical \& Clinical Care Providers}
    \lipsum[1][1]
    \subsubsection{GP \& Specialist Care}
    \lipsum[1][1]
    \subsubsection{Industry (Pharma / Medical Products)}
    \lipsum[1][1]
    \subsubsection{Patients \& Society}
    \lipsum[1][1]
    \subsection{Drug Discovery}
    \subsubsection{Drug Discovery Approaches}
    \subsubsection{AI in Drug Discovery}
    \lipsum[1][1]
    Test

    \subsection{Personalized Care}
    \subsubsection{Medication Monitoring}
    \subsubsection{Virtual Nursing Assistants}

    \subsection{Blockchain in Healthcare}
    \subsubsection{Introduction to Blockchains \& Medical Blockchains}
    \subsubsection{Blockchain in Organ Procurement}
    \subsubsection{Blockchain for Electronic Health Records (EHR)}
    \subsubsection{Blockchain for Pharma Supply Chain (forged drugs, etc.)}

    \subsection{Fraud Detection}
    \subsubsection{Introduction to Fraud Detection}
    \subsubsection{ICD-10 Codes}
    \subsubsection{Fraud Detection in Healthcare Management}

    \subsection{Regulations and Ethics}
    \subsubsection{Legal \& Regulatory Requirements}
    \subsubsection{Data Protection Foundations, GDPR}
    \subsubsection{Privacy in Machine Learning and AI}
    \subsubsection{Bias \& Fairness in AI}
    \subsubsection{Explainable AI}
    Test

    \section{AI in Medical Imaging and Diagnostics}
    \subsection{Introduction to Medical Imaging and Diagnostics}
    \subsubsection{History of Image-Based Diagnostics}
    \subsubsection{Obtaining Ground-Truth Data}
    \subsubsection{Domain Expertise \& Integration into Clinical Practice}
    \subsubsection{Explainability \& Bias in Medical AI}

    \subsection{Medical Imaging Techniques}
    \subsubsection{X-Ray and Computer Aided Tomography (CT)}
    \subsubsection{Magnetic Resonance Imaging (MRI)}
    \subsubsection{Positron Emission Tomography (PET)}
    \subsubsection{Ultrasound Imaging}

    \subsection{Computer Vision Fundamentals}
    \subsubsection{Low Level Computer Vision}
    \subsubsection{Mid Level Computer Vision}
    \subsubsection{High Level Computer Vision}
    Test

    \subsection{Computer Vision with Deep Learning}
    \subsubsection{Image Classification}
    \subsubsection{Object Detection}
    \subsubsection{Image Segmentation}
    \subsubsection{Further Topics}

    \subsection{Applications of AI in Medical Imaging \& Case Studies}
    \subsubsection{Disease Identification}
    \subsubsection{Image Acquisition}
    \subsubsection{Survival Prediction}

    \section{Medical NLP}
    \subsection{Introduction to NLP}
    \subsubsection{Human Language and Meaning of Words}
    \subsubsection{Challenges in NLP}
    \subsubsection{Bias}
    \subsubsection{Evaluation Metrics}
    \subsection{Language Modeling and Word Representation}
    \subsubsection{N-Grams}
    \subsubsection{Bag of Words and Word Vectors}
    \subsubsection{Word Embedding Models}
    \subsection{NLP with Deep Learning}
    \subsubsection{Recurrent Neural Network based Approaches}
    \subsubsection{Transformer based Approaches}
    \subsection{NLP Tasks}
    \subsubsection{Named Entity Recognition}
    \subsubsection{Sentiment Analysis}
    \subsubsection{Text Summarization}
    \subsubsection{Machine Translation}
    \subsubsection{Speech Recognition \& Synthesis}
    \subsubsection{Text Understanding \& Information Extraction}
    \subsection{Application Scenarios \& Case Studies}
    \subsubsection{Medical Text Analysis}
    \subsubsection{Medical Chatbots}
    \subsubsection{Diagnostics and Therapy}
    \subsubsection{Drug Discovery}

    \section{Medical Robotics and Devices}
    \subsection{Internet of Medical Things}
    \subsubsection{Medical Robots}
    \lipsum[1][1]
    \subsubsection{Data-Driven Medicine}
    \subsubsection{Image Management}
    \subsubsection{Cybersecurity}
    \subsubsection{Current Legislation and Trends}
    \subsection{Weareable and Implantable Medical Devices}
    \subsubsection{Weareable Devices}
    \subsubsection{Weareable Sensors for Monitoring}
    \subsubsection{Implantable Devices}
    \subsection{Fundamentals of Robotics: Kinematics}
    \subsubsection{Kinematics}
    \subsubsection{Position and Orientation of a Rigid Body}
    \subsubsection{Joint Kinematics}
    \subsubsection{Forward Kinematics}
    \subsubsection{Inverse Kinematics}
    \subsubsection{Differential Kinematics}
    \subsection{Navigation and Registration}
    \subsubsection{Digitally Reconstructed Radiographs}
    \subsubsection{Points and Landmarks}
    \subsubsection{Contour-Based Registration}
    \subsubsection{Intensity-Based Registration}
    \subsubsection{Image Deformation}
    \subsubsection{Hand-Eye Calibration}
    \subsection{Treatment Planning}
    \subsubsection{Orthopedic Surgery}
    \subsubsection{Radiosurgery}
    \subsubsection{Four-Dimensional Planning}
    \subsection{Design of Medical Robots}
    \subsubsection{Kinematics and Dynamics}
    \subsubsection{Design Methods}
    \subsubsection{Actuators, Sensors, and Material}
    \subsubsection{Security and Safety}

    \section{Literature}
    \begin{itemize}
        \item Alleyn, T. et al (eds) (2020): Target Discovery and Validation: Methods and Strategies for Drug Discovery, Wiley.
        \item Antonopoulos, A. (2017): Mastering Bitcoin, 2nd ed. O’Reilley.
        \item Blass, B. (2015). Basic Principles of Drug Discovery and Development, Academic Press.
        \item Boccia, S. et al. (eds.) (2020): Personalised Health Care, Springer.
        \item Brown, N. (2020): Artificial Intelligence in Drug Discovery, Royal Society of Chemistry.
        \item Challen, R. et al. (2019). Artificial intelligence, bias and clinical safety. BMJ Quality \& Safety, 28(3), 231-237.
        \item Costigliola, V. ed. (2012): Healthcare Overview: New Perspectives, Springer.
        \item Denton, B. (2013): Handbook of Healthcare Operations Management: Methods and Applications, Springer.
        \item Gupta, D. et al. (eds) (2020): Advanced Computational Intelligence Techniques for Virtual Reality in Healthcare, Springer.
        \item Hall, R. (ed) (2012): Handbook of Healthcare System Scheduling, Springer.
        \item IT Governance Privacy Team (2020): EU General Data Protection Regulation (GDPR) – An implementation and compliance guide, fourth edition, ITGP.
        \item Lantz, L., Cawrey, D. (2020): Mastering Blockchain, O’Reilley.
        \item Levine, A. et al. (eds) (2013) The Comprehensive Textbook of Healthcare Simulation, Springer.
        \item McCradden, M. D. et al. (2020). Patient safety and quality improvement: Ethical principles for a regulatory approach to bias in healthcare machine learning. Journal of the American Medical Informatics Association, 27(12), 2024-2027.
        \item Molnar, Ch. (2019), Interpretable Machine Learning, Lulu.
        \item O’Donnell, J. et al (eds.) (2021): Drug Discovery and Development, 3rd ed., CRC Press.
        \item Pinedo, M. (201): Scheduling: Theory, Algorithms, and Systems, 5th ed., Springer.
        \item Vissers, J. (2005): Health Operations Management.
        \item Banerjee, I. et al. (2021): Reading Race: AI Recognizes Patient’s Racial Identity In Medical Images, arXiv preprint https://arxiv.org/abs/2107.10356.
        \item Bushberg, J. et al (2020): The Essential Physics of Medical Imaging, 4th ed. Wolters Kluwer Health.
        \item Esteva, A. et al. Dermatologist-level classification of skin cancer with deep neural networks. Nature 542, 115–118 (2017). https://doi.org/10.1038/nature21056.
        \item Feeman, T. (2015): The Mathematics of Medical Imaging, 2nd ed., Springer.
        \item Forsyth, D., Ponce, J. (2012): Computer Vision - A Modern Approach, Prentice Hall.
        \item Geron, A. (2019), Hands-On Machine Learning with Scikit-Learn, Keras, and TensorFlow, 2nd ed., O’Reilley.
        \item Goodfellow et al (2016): Deep Learning, MIT Press.
        \item Oren, O., Gersh, B., Bhatt, D. (2020): Artificial intelligence in medical imaging: switching from radiographic pathological data to clinically meaningful endpoints, The Lancet, VOLUME 2, ISSUE 9, E486–E488, DOI: https://doi.org/10.1016/S2589-7500(20)30160-6.
        \item Poldrack, R. et al (2011): Handbook of Functional MRI Data Analysis, Cambridge University Press.
        \item Roberts, M., Driggs, D., Thorpe, M. et al. (2021) Common pitfalls and recommendations for using machine learning to detect and prognosticate for COVID-19 using chest radiographs and CT scans. Nat Mach Intell 3, 199–217. https://doi.org/10.1038/s42256-021-00307-0.
        \item Smith, N., Webb, A. (2010): Introduction to Medical Imaging, Cambridge University Press.
        \item Stippich, Ch. (ed) (2021): Clinical Functional MRI, Springer.
        \item Szeliski, R. (2010): Computer Vision - Algorithms and Applications, Springer.
        \item Wang, S. et al (2016): Accelerating magnetic resonance imaging via deep learning, 2016 IEEE 13th International Symposium on Biomedical Imaging (ISBI).
        \item Wulczyn, E. et al (2021): Interpretable survival prediction for colorectal cancer using deep learning, npj Digit. Med. 4, 71, DOI: https://doi.org/10.1038/s41746-021-00427-2.
        \item Wynants, L. et al (2020), Prediction models for diagnosis and prognosis of covid-19: systematic review and critical appraisal, BMJ 2020;369:m1328.
        \item Clark, A., Fox, C., \& Lappin, S. (Eds.). (2010). Handbook of computational linguistics and natural language processing. Malden, MA: Wiley-Blackwell.
        \item Devlin, J. et al (2018). Bert: Pre-training of deep bidirectional transformers for language understanding. arXiv preprint arXiv:1810.04805.
        \item Ethayarajh, K. (2020). Is Your Classifier Actually Biased? Measuring Fairness under Uncertainty with Bernstein Bounds. Proceedings of the 58th Annual Meeting of the Association for Computational Linguistics.
        \item Fraser, K.C., Meltzer, J., \& Rudzicz, F. (2016). Linguistic Features Identify Alzheimer's Disease in Narrative Speech. Journal of Alzheimer's disease : JAD, 49(2), 407--422.
        \item Garrido-Muñoz, I., et al. (2021). A Survey on Bias in Deep NLP. Applied Sciences, 11(7):3184. https://doi.org/10.3390/app11073184
        \item Juhn, Y., \& Liu, H. (2020). Artificial intelligence approaches using natural language processing to advance EHR-based clinical research. Journal of Allergy and Clinical Immunology, 145(2), 463--469.
        \item Kandpal, P., et al. (2020). Contextual Chatbot for Healthcare Purposes (using Deep Learning). Fourth World Conference on Smart Trends in Systems, Security and Sustainability (WorldS4), 2020, pp. 625--634. doi:10.1109/WorldS450073.2020.9210351
        \item Martin, J., \& Jurafsky, D. (2020). Speech and Language Processing, 3rd ed., Prentice Hall.
        \item Masrani, V., et al. (2017). Domain Adaptation for Detecting Mild Cognitive Impairment. Canadian Conference on AI.
        \item Murray, G. (2018). Language-Based Automatic Assessment of Cognitive and Communicative Functions Related to Parkinson’s Disease.
        \item Schick, T., Udupa, S., \& Schütze, H. (2021). Self-diagnosis and self-debiasing: A proposal for reducing corpus-based bias in NLP. arXiv preprint arXiv:2103.00453.
        \item Sorin, V., et al. (2020). Deep Learning for Natural Language Processing in Radiology—Fundamentals and a Systematic Review. Journal of the American College of Radiology, 17(5), 639--648.
        \item Sun, T., et al. (2019). Mitigating Gender Bias in Natural Language Processing: Literature Review. arXiv pre-print 1906.08976.
        \item Tay, Y., et al. (2020). Efficient transformers: A survey. arXiv preprint arXiv:2009.06732.
        \item Vaswani, A., et al. (2017). Attention is all you need. In Advances in Neural Information Processing Systems (pp. 5998--6008).
        \item Zand, A., et al. (2020). An Exploration Into the Use of a Chatbot for Patients With Inflammatory Bowel Diseases: Retrospective Cohort Study. Journal of Medical Internet Research, 22(5), e15589. https://doi.org/10.2196/15589
        \item Cardona, M., Solanki, V. K., \& Garcia Cena, C. E. (Eds.). (2021). Internet of Medical Things. Boca Raton: CRC Press.
        \item Schweikard, A., \& Ernst, F. (2015). Medical Robotics. Springer International Publishing. https://doi.org/10.1007/978-3-319-22891-4
        \item Troccaz, J. (Ed.). (2013). Medical Robotics. John Wiley and Sons. https://doi.org/10.1002/9781118562147
    \end{itemize}

    % ==== Natural Language Processing and Voice Assistants =============================================================================
    \chapter{Natural Language Processing and Voice Assistants}
    \section{Natural Language Processing}
    \subsection{Introduction to NLP}
    \subsubsection{What is NLP?}
    \lipsum[25][1-3]
    \subsubsection{Syntax, Semantics and Prosodics}
    \lipsum[26][1-3]
    \subsubsection{Phonetics and Speech}
    \lipsum[1][1]
    \subsubsection{Evaluation of NLP Systems}
    \lipsum[1][1]
    \subsection{Text Processing}
    \subsubsection{Word Vectors and Word Embeddings}
    \subsubsection{Regular Expressions}
    \subsubsection{Statistical Approaches}
    \subsubsection{Recurrent Neural Network based Approaches}
    \subsubsection{Transformer based Approaches}
    \subsection{Speech Processing}
    \subsubsection{Statistical Speech Recognition and Synthesis}
    \subsubsection{Speech Recognition and Synthesis with Deep Learning}
    \subsection{Application Scenarios}
    \subsubsection{Speech Recognition, Speech Synthesis and Machine Translation}
    \subsubsection{Information Extraction and Text Understanding}
    \subsubsection{Chatbots and Voice Assistants}
    \subsubsection{NLP in Education}
    \lipsum[26][1-3]

    \subsubsection{NLP with Python}
    \subsection{Challenges in NLP}
    \subsubsection{Data for NLP}
    \subsubsection{Domain and Language Adaptation}
    \subsubsection{Explainability}
    \lipsum[26][1-3]

    \subsubsection{Bias}
    \section{NLP in Education}
    \section{NLP for Accessibility}

    \section{Literature}
    \begin{itemize}
        \item Bird, S., Klein, E., \& Loper, E. (2009). Natural Language Processing with Python. O'Reilly.
        \item Jurafsky, D., \& Martin, J. H. (2020). Speech and Language Processing (3rd ed.). Prentice Hall. \url{https://web.stanford.edu/~jurafsky/slp3}
        \item Kamath, U., Liu, J., \& Whitaker, J. (2019). Deep Learning for NLP and Speech Recognition: Practical NLP, Speech, and Deep Learning using Python-based Open Source Tools. Springer.
        \item Bocklisch, T., Faulker, J., Pawlowski, N., \& Nichol, A. (2017). Rasa: Open Source Language Understanding and Dialogue Management. NIPS Workshop on Conversational AI.
        \item Chen, L., Chen, P., \& Lin, Z. (2020): Artificial Intelligence in Education: A Review. IEEE Access 8 (2020), 75264--75278.
        \item Heffernan, N. (2014): The ASSISTments Ecosystem: Building a Platform that Brings Scientists and Teachers Together for Minimally Invasive Research on Human Learning and Teaching. International Journal of Artificial Intelligence in Education 24 (12 2014).
        \item Libbrecht, P., Declerck, T., Schlippe, T., Mandl, T., \& Schiffner, D. (2020): NLP for Student and Teacher: Concept for an AI based Information Literacy Tutoring System. In The 29th ACM International Conference on Information and Knowledge Management (CIKM2020). Galway, Ireland.
        \item Schlippe, T., \& Sawatzki, J. (2021): AI-based Multilingual Interactive Exam Preparation. In The Learning Ideas Conference 2021 (14th annual conference). ALICE - Special Conference Track on Adaptive Learning via Interactive, Collaborative and Emotional Approaches. New York, USA.
        \item Schlippe, T., \& Sawatzki, J. (2021): Cross-Lingual Automatic Short Answer Grading. In Proceedings of The 2nd International Conference on Artificial Intelligence in Education Technology (AIET 2021). Wuhan, China.
        \item Al-Thanyyan, S. S., \& Azmi, A. (2021): Automated Text Simplification: A Survey. ACM Computing Surveys, Vol. 54, Issue 2, pp. 1--36.
        \item Klaper, D., Ebling, S., \& Volk, M. (2013): Building a German/Simple German Parallel Corpus for Automatic Text Simplification. The Second Workshop on Predicting and Improving Text Readibility for Target Reader Populations.
        \item Schlippe, T., Alessai, S., El-Taweel, G., Wölfel, M., \& Zaghouani, W. (2020): Visualizing Voice Characteristics with Type Design in Closed Captions for Arabic. Cyberworlds. Caen, France.
    \end{itemize}

    % ==== Industrial AI =============================================================================
    \chapter{Industrial AI}
    \section{AI in Production}
    \subsection{Introduction: The Smart Factory}
    \subsubsection{Goals of a Smart Factory}
    \lipsum[27][1-3]
    \subsubsection{Internet of Things}
    \lipsum[28][1-3]
    \subsubsection{Cyber-Physical Systems}
    \subsubsection{Cyber-Physical Production Systems}
    \lipsum[1][1]
    \subsubsection{A New Paradigm for Automation}
    \lipsum[1][1]
    \subsection{Basics of a Smart Factory}
    \subsubsection{Intelligent Products, Object Identification and Digital Object Memory}
    \subsubsection{Formal Languages and Ontologies}
    \subsubsection{Autonomous Cooperation}
    \lipsum[1][1]
    \subsubsection{Humans \& Machines}
    \subsubsection{Order-Controller Production}
    \lipsum[1][1]
    \subsubsection{Smart Services}
    \subsection{AI for Design}
    \subsubsection{Generative Design}
    \subsubsection{Methods}
    \subsection{AI for Quality}
    \subsubsection{Fault Detection \& Identification}
    \subsubsection{Predictive and Prescriptive Maintenance}
    \subsubsection{Defect Recognition}
    \subsection{AI for Supply Chain}
    \lipsum[26][1-3]

    \subsubsection{Demand Forecasting}
    \subsubsection{Inventory Models}
    \subsection{AI for Autonomous Planning and Scheduling}
    \subsubsection{Introduction}
    \subsubsection{Methods}

    \section{Industrial Automation}
    \subsection{Introduction to Production Systems}
    \subsubsection{Basic concepts and definitions}
    \subsubsection{Industrial supervision and control}
    \subsubsection{Challenges}
    \subsubsection{Trends}

    \subsection{Automata}
    \subsubsection{Preliminaries}
    \subsubsection{Deterministic finite automata}
    \subsubsection{Non-deterministic finite automata}
    \subsubsection{Properties}

    \subsection{Petri nets}
    \subsubsection{Preliminaries}
    \subsubsection{Modeling systems}
    \subsubsection{Properties}
    \subsubsection{Analysis methods}
    \lipsum[26][1-3]

    \subsection{Timed models}
    \subsubsection{Timed automata}
    \subsubsection{Markov processes}
    \subsubsection{Queuing theory}
    \subsubsection{Timed Petri nets}

    \subsection{Simulation of discrete event systems}
    \subsubsection{Basic concepts}
    \subsubsection{Working principles}
    \subsubsection{Performance analysis}
    \subsubsection{Software tools}
    \lipsum[26][1-3]

    \subsection{Supervisory control}
    \subsubsection{Basic concepts}
    \subsubsection{Specifications}
    \subsubsection{Synthesis}
    \subsubsection{Performance analysis}
    \subsubsection{Implementation}

    \subsection{Applications}
    \subsubsection{Production system supervision}
    \subsubsection{Monitoring and diagnosis of faults}
    \subsubsection{Distributed and de-centralized supervision}
    \subsubsection{Model-based optimization of production systems}
    \subsubsection{Adaptive supervisory control}
    \lipsum[26][1-3]

    \section{Industrial and Mobile Robots}
    \subsection{Introduction}
    \subsubsection{Robots and manufacturing}
    \subsubsection{Industrial robots}
    \subsubsection{Mobile robots}
    \subsubsection{Actuators for robotics}
    \subsubsection{Trends in robotics}

    \subsection{Kinematics}
    \subsubsection{Position and orientation of a rigid body}
    \subsubsection{Joint kinematics}
    \subsubsection{Forward kinematics}
    \subsubsection{Inverse kinematics}
    \subsubsection{Differential kinematics}
    \subsubsection{Kinematics of mobile robots}
    \lipsum[26][1-3]

    \subsection{Trajectory Planning}
    \subsubsection{Basic concepts}
    \subsubsection{Trajectories in the joint space}
    \subsubsection{Trajectories in the workspace}
    \subsubsection{Trajectory planning for mobile robots}

    \subsection{Sensing and Perception}
    \subsubsection{Position}
    \subsubsection{Velocity}
    \subsubsection{Force}
    \subsubsection{Distance}
    \subsubsection{Visual}

    \subsection{Fundamentals of Robot Dynamics}
    \subsubsection{Rigid body dynamics}
    \subsubsection{Lagrange formulation}
    \subsubsection{Newton formulation}
    \subsubsection{Direct and inverse dynamics}
    \subsubsection{Dynamics of mobile robots}
    \lipsum[26][1-3]

    \subsection{Control of Robots}
    \subsubsection{Basic concepts}
    \subsubsection{Decentralized motion control}
    \subsubsection{Centralized motion control}
    \subsubsection{Force control}
    \lipsum[26][1-3]

    \subsection{Architecture of Robotic Systems}
    \subsubsection{Architectural components}
    \subsubsection{Open Robot Control Software (OROCOS)}
    \subsubsection{Yet Another Robotic System Platform (YARP)}
    \subsubsection{Robot Operating System (ROS)}
    \subsubsection{Behavior-based robotics}

    \section{Literature}
    \begin{itemize}
        \item Dafflon, B. Moalla, N., \& Ouzrout, Y. (2021). The challenges, approaches, and used techniques of CPS for manufacturing in Industry 4.0: a literature review. The International Journal of Advanced Manufacturing Technology, 113(7), 2395--2412. https://doi.org/10.1007/s00170-020-06572-4
        \item Mahmood, Z. (Ed.). (2019). The Internet of Things in the Industrial Sector. Cham: Springer International Publishing. https://doi.org/10.1007/978-3-030-24892-5
        \item Manesis, S., \& Nikolakopoulos, G. (2020). Introduction to industrial automation. CRC Press, Taylor \& Francis Group.
        \item Singh, K. K., Nayyar, A., Tanwar, S., \& Abouhawwash, M. (Eds.). (2021). Emergence of Cyber Physical System and IoT in Smart Automation and Robotics. Cham: Springer International Publishing. https://doi.org/10.1007/978-3-030-66222-6
        \item Veneri, G., \& Capasso, A. (2018). Hands-On Industrial Internet of Things. Packt. Retrieved from https://www.packtpub.com/product/hands-on-industrial-internet-of-things/9781789537222
        \item Cassandras, C. G., \& Lafortune, S. (2009). Introduction to discrete event systems. Springer.
        \item Hooley, G., Nicoulaud, B., Rudd, J. M., \& Piercy, N. (2019). Marketing strategy and competitive positioning. Pearson.
        \item Kaplan, R., Norton, D., \& Rugelsjoen, B. (2010). Managing alliances with the balanced scorecard. Harvard Business Review, 88(1/2), 114--120.
        \item Linz, P. (2006). An introduction to formal languages and automata. Jones \& Bartlett Learning.
        \item Reisig, W. (2013). Understanding Petri nets: Modeling techniques, analysis methods, case studies. Springer.
        \item Stewart, J. B. (2013, October 14). The collapse: How a top legal firm destroyed itself. The New Yorker.
        \item Ben-Ari, M., \& Mondada, F. (2017). Elements of robotics. Springer International Publishing.
        \item Siciliano, B., Sciavicco, L., Villani, L., \& Oriolo, G. (2009). Robotics. Springer.
        \item Siciliano, B., \& Khatib, O. (Eds.). (2016). Springer handbook of robotics (2nd ed.). Springer.
    \end{itemize}

    % ==== Artificial Intelligence in Supply Chain Management =============================================================================
    \chapter{Artificial Intelligence in Supply Chain Management}
    \section{Concepts of Artificial Intelligence in Supply Chain Management}
    \subsection{Fundamentals of Supply Chain Management}
    \subsubsection{Concept of Supply Chain and Supply Network}
    \lipsum[29][1-3]
    \subsubsection{End-to-End View of Supply Chain Management}
    \lipsum[30][1-3]
    \subsubsection{The Vision of Supply Chain 4.0}
    \subsection{Conceptional and Mathematical Introduction to Key Artificial Intelligence Disciplines for Supply Chains}
    \subsubsection{Conventional Techniques}
    \subsubsection{Machine Learning Algorithms}
    \subsubsection{Neural Networks}
    \subsubsection{Robot Process Automation}
    \subsubsection{Multi-Agent Systems}
    \subsection{Models for improving transparency along Supply Chains}
    \subsubsection{Customer and Churn Analytics}
    \lipsum[41][1-3]
    \subsubsection{Order Peak time Prediction}
    \subsubsection{Risk and Fraud Detection}
    \subsubsection{Spend Analytics}
    \subsubsection{Defect Detection and Predictive Maintenance}
    \subsection{Methods to Support Strategic and Tactical Decision-Making in Supply Chains}
    \subsubsection{Supply Chain Network Planning}
    \subsubsection{Supplier Selection}
    \subsubsection{Replenishment Strategies}
    \subsubsection{Route Optimization}
    \subsubsection{Sales \& Operations Planning}
    \lipsum[26][1-3]

    \subsection{AI Concepts in Supply Chain Operations}
    \subsubsection{Supplier Communication and Purchasing}
    \subsubsection{Autonomous Allocation of Orders to Production Resources}
    \subsubsection{Dynamic Routing}
    \subsubsection{Object Identification in Logistics}
    \subsection{Challenges of applying AI in Supply Chains}
    \subsubsection{The Challenge of Trust}
    \subsubsection{The Challenges of Capability}
    \subsubsection{The Challenges of Accountability}
    \subsubsection{The Challenges of Accessibility}
    \subsubsection{The Challenges of Organizational Transformation}
    \lipsum[26][1-3]

    \section{Demand Forecast and Inventory Control}
    \subsection{Newsvendor Model}
    \subsubsection{Single Period Newsvendor (Classic, Cost Function)}
    \subsubsection{Demand as a Stochastic Quantity}
    \subsubsection{Demand Models}
    \subsubsection{Handling Censored Data}
    \subsubsection{Extensions}
    \subsubsection{Multi-Period Newsvendor}
    \lipsum[42][1-3]

    \subsection{Traditional Methods of Demand Forecasting}
    \subsubsection{Exponential Smoothing}
    \subsubsection{ARIMA}
    \subsubsection{State Space Models}
    \subsubsection{(Bayesian) Structural Time Series Models}
    \subsection{Data Driven Methods for Demand Forecasting}
    \subsubsection{Recurrent Neural Networks}
    \subsubsection{Supervised Learning}
    \subsubsection{Effects of Correlation and Confounding}
    \subsubsection{Big Data Newsvendor}
    \subsection{Inventory Models}
    \subsubsection{Economic Order Quantity}
    \subsubsection{Inventory Models with Review}
    \subsubsection{Inventory Models with Service Levels}
    \lipsum[26][1-3]

    \subsection{Further Effects}
    \subsubsection{Customer Heterogeneity}
    \subsubsection{Finite Product Lifetime}
    \subsubsection{Minimum Order Quantity}
    \subsubsection{Delivery Schedules}
    \subsubsection{Operational KPIs and Inventory Optimization}

    \section{Literature}
    \begin{itemize}
        \item Chopra, S. (2019): Supply chain management. Strategy, planning, and operation. Pearson.
        \item Hwang, I. \& Jang, Y. J. (2020): Q($\lambda$) learning-based dynamic route guidance algorithm for overhead hoist transport systems in semiconductor fabs, International Journal of Production Research, 58:4, 1199-1221. DOI: 10.1080/00207543.2019.1614692
        \item Masrour, T., Cherrafi, A., El Hassani, I. (2021): Artificial Intelligence and Industrial Applications: Smart Operation Management, Springer
        \item Park, J.; Kim, M.-H.; Choi, D.-G. (2021): Correspondence Learning for Deep Multi-Modal Recognition and Fraud Detection. Electronics 2021, 10, 800. https://doi.org/10.3390/electronics10070800
        \item Selvakanmani, S., Pranamita, N., Deepak, K., Kavi, B.A., Salmaan, A.K. (2020): Churn prediction using ensemble learning: an analytical CRM application. Int. J. Adv. Sci. Technol. Vol. 29, No. 5, p. 9192--9200
        \item Triepels, R., Daniels, H., Feelders, A. (2019): Data-driven fraud detection in international shipping, Expert Systems With Applications 99, Elsevir, p.193-202.
        \item Wang, C. \& Jiang, P (2019). Deep neural networks based order completion time prediction by using real-time job shop RFID data. Journal of Intelligent Manufacturing, Vol. 30, No. 3, p. 1303--1318.
        \item Ban GY, Rudin C (2019) The big data newsvendor: Practical insights from machine learning. Oper Res 67(1):90--108
        \item Bertsimas D, Kallus N (2020) From predictive to prescriptive analytics. Manag Sci 66(3):1025--1044
        \item Bica I, Alaa A, Van Der Schaar M (2020) Time series deconfounder: Estimating treatment effects over time in the presence of hidden confounders. In: International Conference on Machine Learning. PMLR, pp 884--895
        \item Brodersen KH, Gallusser F, Koehler J, Remy N, Scott SL (2015) Inferring causal impact using Bayesian structural time-series models. Ann Appl Stat 9:247--274
        \item De Gooijer JG, Hyndman RJ (2006) 25 years of time series forecasting. Int J Forecast 22(3):443--473
        \item Galliher, H. P., Morse, Philip M., Simond, M. (1959) Dynamics of two classes of continuous-review inventory systems. Operations Research, 7(3):362--384
        \item Harris, F. (1990) How many parts to make at once. Operations Research, 38(6):947--950.
        \item Hillier, F., Liebermann, G. (2020), ISE Introduction to Operations Research, McGraw-Hill
        \item Huber J, M\"uller S, Fleischmann M, Stuckenschmidt H (2019) A data-driven newsvendor problem: From data to decision. Eur J Oper Res 278(3):904--915
        \item Hyndman R, Koehler AB, Ord JK, Snyder RD (2008) Forecasting with exponential smoothing: the state space approach. Springer Science \& Business Media
        \item Khouj, M (1999) The single-period (news-vendor) problem: literature review and suggestions for future research. Omega 27(5):537--553
        \item L\"angkvist M, Karlsson L, Loutfi A (2014) A review of unsupervised feature learning and deep learning for time-series modeling. Pattern Recogn Lett 42:11--24
        \item Lim B, Arik SO, Loeff N, Pfister T (2019) Temporal fusion transformers for interpretable multi-horizon time series forecasting. arXiv preprint arXiv:1912.09363
        \item Malinsky D, Spirtes P (2018) Causal structure learning from multivariate time series in settings with unmeasured confounding. In: Proceedings of 2018 ACM SIGKDD Workshop on Causal Discovery, pp 23--47
        \item Nahmias, S., Pierskalla, W. (1973) Optimal ordering policies for a product that perishes in two periods subject to stochastic demand. Naval Research Logistics Quarterly, 20(2):207--229, 1973.
        \item Porteus, E. (1983) Inventory Policies for Periodic Review Systems. Research Papers 650, Stanford University, Graduate School of Business, June 1983.
        \item Rasul K, Sheikh AS, Schuster I, Bergmann U, Vollgraf R (2020) Multi-variate probabilistic time series forecasting via conditioned normalizing flows. arXiv preprint arXiv:2002.06103
        \item Runge J (2018) Causal network reconstruction from time series: From theoretical assumptions to practical estimation. Chaos Int J Nonlinear Sci 28(7):075310
        \item Scarf, H. (1959) Bayes solutions of the statistical inventory problem. Ann. Math. Statist., 30(2):490--508, 06
        \item Vandeput, N. (2020) Inventory Optimization: Models and Simulations, De Gruyter
        \item Wick, F. et al. (2021) Demand Forecasting of Individual Probability Density Functions with Machine Learning. SN Oper. Res. Forum 2, 37. https://doi.org/10.1007/s43069-021-00079-8
        \item Cavalcante, I. M., Frazzon, E. M., Forcellini, F. A., Ivanov, D. (2019): A supervised machine
              learning approach to data-driven simulation of resilient supplier selection in digital
              manufacturing, International Journal of Information Management, vol. 49, p. 86--97, \url{https://doi.org/10.1016/j.ijinfomgt.2019.03.004}.
        \item Chopra, A. (2019): AI in Supply \& Procurement. Amity International Conference on Artificial
              Intelligence (AICAI), February 2019: 308--316. \url{https://doi.org/10.1109/AICAI.2019.8701357}.
        \item Krebs, K. D. (2020). How Can the DoD Adopt Commercial-Style Artificial Intelligence for
              Procurement? Artificial-Intelligence Capabilities, NPS-CM-20-152, \url{https://dair.nps.edu/handle/123456789/4159}.
        \item McGroarty, F., Booth, A., Gerding, E., \& Chinthalapati, V. L. R. (2019). High frequency trading
              strategies, market fragility and price spikes: an agent based model perspective. Annals of
              Operations Research, 282(1/2), 217--244. \url{https://doi-org.pxz.iubh.de:8443/10.1007/s10479-018-3019-4}.
        \item Mitchell, B., \& Trebes, B. (2018). Procurement. In Managing Reality (3rd Edition) - Complete Set. ICE Publishing.
        \item Marquis, P., Papini, O., \& Prade, H. (2020). A Guided Tour of Artificial Intelligence
              Research: Volume II: AI Algorithms. Springer.
        \item Sivakriskul, K., \& Phienthrakul, T. (2021): Product Category Recommendation System Using
              Markov Model. In: Kaiser M.S., Xie J., Rathore V.S. (eds) Information and Communication
              Technology for Competitive Strategies (ICTCS 2020). Lecture Notes in Networks and Systems,
              vol. 190. Springer, Singapore. \url{https://doi-org.pxz.iubh.de:8443/10.1007/978-981-16-0882-7_60}.
        \item Torres Berru, Y., López Batista, V. F., Torres-Carrión, P., Jimenez, M. G. (2020): Artificial Intelligence
              Techniques to Detect and Prevent Corruption in Procurement: A Systematic Literature Review.
              In: Botto-Tobar M., Zambrano Vizuete M., Torres-Carrión P., Montes León S., Pizarro Vásquez G.,
              Durakovic B. (eds) Applied Technologies. ICAT 2019. Communications in Computer and
              Information Science, vol. 1194. Springer, Cham. \url{https://doi-org.pxz.iubh.de:8443/10.1007/978-3-030-42520-3_21}.
        \item Zhou, R., Pang, J., Wang, Z., Lui, J. C. S., \& Li, Z. (2021): A Truthful Procurement Auction for
              Incentivizing Heterogeneous Clients in Federated Learning. 2021 IEEE 41st International
              Conference on Distributed Computing Systems (ICDCS), 183--193. \url{https://doi-org.pxz.iubh.de:8443/10.1109/ICDCS51616.2021.00026}.
    \end{itemize}

    % ==== Multi-Agent Systems =============================================================================
    \chapter{Multi-Agent Systems}
    \section{Agent technology}
    \subsection{Concept of Agents and Multi-Agent Systems}
    \lipsum[31][1-3]
    \subsection{Agent Applications}
    \lipsum[32][1-3]
    \subsection{Agents Oriented Design and Methodologies}
    \section{Typology of Intelligent Agents}
    \subsection{Reasoning Agents}
    \lipsum[26][1-3]

    \subsection{Reactive Agents}
    \subsection{Hybrid Agents}
    \section{Agent Communication}
    \subsection{Ontology}
    \subsection{Communication Languages}
    \lipsum[26][1-3]

    \section{Agent Cooperation}
    \subsection{Distributed Problem Solving}
    \subsection{Task and Result Sharing}
    \subsection{Handling Inconsistency}
    \subsection{Planning and Synchronization}
    \section{Multi-Agent Decision-Making}
    \lipsum[26][1-3]

    \subsection{Strategies}
    \subsection{Group Decisions}
    \subsection{Coalitions}
    \subsection{Bargaining}
    \subsection{Arguing}
    \section{Reinforcement Learning-Multi-Agent}
    \subsection{The Goal of Reinforcement Learning}
    \lipsum[26][1-3]

    \subsection{Benefits and Challenge}
    \subsection{Introducing Multi-Agent Reinforcement Learning Algorithms}
    \section{Potentials of Multi-Agent Applications in Supply Chains}
    \subsection{Multi-Agents Application for Strategic and Tactical Tasks}
    \subsection{Multi-Agents Application in Operational Processes}
    \subsection{Multi-Agents Embedded in Cyber-Physical Systems}

    \section{Literature}
    \begin{itemize}
        \item Bellifemine, F. L., Caire, G., Greenwood, D. (2007): Developing Multi-Agent Systems with JADE. Wiley.
        \item Bordini, R. H., Dastani, M., Dix, J., El Fallah Seghrouchni, A. (2009): Multi-Agent Programming, Languages, Tools and Applications. Springer.
        \item Bordini, R., H\"ubner, \& J. F., Wooldridge, M. (2007): Programming Multi-Agent Systems in AgentSpeak using Jason. Wiley.
        \item Paolucci M, Sacile R (2016) Agent-based manufacturing and control systems: new agile manufacturing solutions for achieving peak performance. CRC Press.
        \item Shoham, Y., \& Leyton-Brown, K. (2009): Multiagent Systems, Algorithmic, Game-Theoretic, and Logical Foundations. Cambridge University Press.
        \item Uhrmacher, A. M., \& Weyns, D. (2009): Multi-Agent Systems, Simulation and Applications. CRC Press.
        \item Weiss, G. (2013): Multiagent Systems, The MIT Press. Cambridge.
    \end{itemize}

    % ==== AI in E-Commerce, Marketing and Demand Forecasting =============================================================================
    \chapter{AI in E-Commerce, Marketing and Demand Forecasting}
    \section{Introduction to AI in E-Commerce and Marketing}
    \subsection{Application Areas and Historical Review}
    \subsubsection{Retail}
    \lipsum[33][1-3]
    \subsubsection{Entertainment}
    \lipsum[34][1-3]
    \subsubsection{Advertising}
    \subsubsection{Internet of Things}
    \subsection{Virtual Assistants}
    \subsubsection{NLP Fundamentals}
    \subsubsection{NLP with Deep Learning}
    \subsubsection{Chatbots}
    \subsubsection{Voice Search}
    \subsection{Visual Search}
    \subsubsection{Computer Vision Fundamentals}
    \subsubsection{Computer Vision with Deep Learning}
    \subsubsection{Visual Product Search}
    \subsection{Dynamic Pricing}
    \lipsum[26][1-3]

    \subsubsection{Pricing Theory}
    \subsubsection{Measuring Price Elasticity}
    \subsubsection{Bayesian Optimal Pricing}
    \subsubsection{Dynamic Pricing}
    \subsection{Regulatory Requirements \& Ethics}
    \subsubsection{Data Protection and Data Privacy}
    \subsubsection{Ethical Data Usage and Modeling}
    \subsection{Case Studies}
    \subsubsection{Retail}
    \subsubsection{Entertainment}
    \lipsum[26][1-3]

    \subsubsection{Advertisement}

    \section{AI in Marketing and Analytics}
    \subsection{Foundations and Introduction}
    \subsubsection{Basic Building Blocks}
    \subsubsection{Channels \& Strategies}

    \subsection{Descriptive Methods}
    \subsubsection{Business Intelligence}
    \subsubsection{Brand Metrics and Value}
    \subsubsection{Customer Segmentation, Journey and Acquisition Cost}
    \subsubsection{Market Basket \& Assortment Analysis}
    \subsubsection{Search Analytics}
    \lipsum[26][1-3]

    \subsection{Predictive Methods}
    \subsubsection{Customer Churn and Retention}
    \subsubsection{Customer Lifetime Value (CLV) Estimation}
    \subsubsection{Sales Forecasting and Budgeting}
    \subsubsection{Search-Optimization}
    \lipsum[26][1-3]

    \subsection{Prescriptive Methods}
    \subsubsection{Pricing Strategies}
    \subsubsection{Upselling \& Cross-selling}
    \subsubsection{Marketing Campaign Analytics and Optimization}
    \subsubsection{Targeting}
    \subsubsection{Marketing Experiments, Tests \& Evaluation}

    \subsection{Perspectives}
    \subsubsection{Closed Loop vs.\ Human-in-the-Loop, Active Learning}
    \subsubsection{Cross-Channel, Omnichannel and Subscriptions}

    \section{Personalization and Recommender Systems}
    \subsection{Foundation and Introduction}
    \subsubsection{History and Application Domains of Recommender Systems}
    \subsubsection{Basic Building Blocks:}
    \subsubsection{Levels of Personalization \& Recommender Archetypes}
    \subsubsection{Business Goals \& Evaluation Strategies}
    \subsection{Collaborative Filtering}
    \subsubsection{Neighborhood-Based Approaches:}
    \subsubsection{Graph-Based Approaches}
    \subsubsection{Latent Factor Models}
    \subsubsection{Bayesian Personalized Ranking (BPR)}
    \subsection{Content-based Filtering}
    \subsubsection{Content Types \& Strategies across Domains}
    \subsubsection{Factorization Machines \& Classification}
    \subsection{Hybrid Recommender Systems}
    \lipsum[26][1-3]

    \subsubsection{User- vs. Item-based Recommendations}
    \subsubsection{Monolithic, Mixed Hybrid and Ensemble Recommenders}
    \subsection{Large-Scale Recommender Systems}
    \subsubsection{Information Retrieval Dichotomy}
    \subsubsection{Approximate Nearest Neighbour Search}
    \subsubsection{Serving Recommendations in Production}
    \subsection{Perspectives}
    \subsubsection{(Contextual) Multi-Armed Bandits}
    \subsubsection{Deep Learning and Reinforcement Learning Based Approaches}
    \subsubsection{Causality-Aware Approaches}
    \subsubsection{Multi-Stakeholder and Multi-Objective Recommender Systems}
    \lipsum[26][1-3]

    \section{Literature}
    \begin{itemize}
        \item Chaffey, D. (2019): Digital Business and E-Commerce Management, 7th ed. Pearson.
        \item Forsyth, D., Ponce, J. (2012): Computer Vision - A Modern Approach, Prentice Hall.
        \item Friedman, D. (1986): Price Theory: An Intermediate Text, South-Western Publishing Co.
        \item IT Governance Privacy Team (2020): EU General Data Protection Regulation (GDPR) -- An implementation and compliance guide, fourth edition, ITGP.
        \item Landsburg, St. (2013): Price Theory and Applications, Cengage Learning; 9th edition.
        \item Laudon, K./Traver, C. G. (2020): E-Commerce. Business. Technology. Society. 16. Auflage, Pearson.
        \item Martin, J., Jurafsky, D. (2008): Speech and Language Processing, 2nd ed., Prentice Hall.
        \item Rogers, D. L. (2016): The digital transformation playbook: Rethink your business for the digital age. Columbia Business School Publishing, New York.
        \item Szeliski, R. (2010): Computer Vision - Algorithms and Applications, Springer, 2010.
        \item Barker, M., Barker, D., Bormann, N. (2016): Social Media Marketing: A Strategic Approach 2nd Edition, Cengage Learning
        \item Butow, E. et al. (2020): Ultimate Guide to Social Media Marketing. Entrepreneur Press, Irvine
        \item Chandler, St. (2012): Own your niche, Authority Publishing
        \item Chaters, B. (2011): Mastering Search Analytics, O’Reilley Publishing
        \item Dib, A. (2018): The 1-Page Marketing Plan, Page Two
        \item Enge, E., Spencer, St., Stricchiola, J. (2015): The Art of SEO, 3rd ed. O’Reilly Media
        \item Grigsby, M.(2018): Marketing Analytics: A Practical Guide to Improving Consumer Insights Using Data Techniques, 2nd Edition, London
        \item Kerin, R. , Hartley, St. (2019): Marketing: The Core, 8th ed., McGraw-Hill
        \item McKinnon, B. (2019): What's Your Point?: The Brand Arrow - Define Your Point. Grow Your Brand, Grace and Down Publishing
        \item Aggarwal, C. (2016): Recommender Systems, Springer.
        \item Falk, K, (2019): Practical Recommender Systems, Manning Publications.
        \item Jannach, D., Zanker, M., Felfernig, A, Friedrich, G. (2010): Recommender Systems: An Introduction, Cambridge University Press.
        \item Moreira, G., Cunha, A. (2020): Deep Learning for News Recommender Systems, LAP LAMBERT Academic Publishing.
        \item Pearl, J., Glymour, M., \& Jewell, N. P. (2016). Causal inference in statistics: A primer. Wiley.
        \item RecSys '20: Fourteenth ACM Conference on Recommender Systems, Association for Computing Machinery, New York, NY, United States, ISBN 78-1-4503-7583-2
        \item Sutton, R. S., \& Barto, A. G. (1998). Reinforcement learning: An introduction. MIT Press.

    \end{itemize}

    % ==== International IT Law =============================================================================
    \chapter{International IT Law}
    \subsection{Introduction}
    \subsubsection{General Concepts of Law}
    \subsubsection{Areas of Law}
    \subsubsection{International, Transnational and EU Law}
    \subsubsection{Definition and Scope of IT Law}
    \subsubsection{International, Transnational and European IT Law}
    \subsubsection{Law in Cross-Border Systems}

    \subsection{E-Business and E-Commerce}
    \subsubsection{General Terms and Conditions of Business}
    \subsubsection{Electronic Commerce}
    \subsubsection{IT Contracts}
    \subsubsection{Intermediaries and Platforms}
    \subsubsection{Antitrust Law and IT}
    \lipsum[26][1-3]

    \subsection{Intellectual Property}
    \subsubsection{Basic Concepts of Intellectual Property}
    \subsubsection{Copyright}
    \subsubsection{Software Copyright and Software Licensing}
    \subsubsection{Free and Open Licensing}
    \subsubsection{Patenting of Software}
    \lipsum[26][1-3]

    \subsection{Privacy and Data Protection}
    \subsubsection{Basic Concepts of Privacy and Data Protection}
    \subsubsection{European General Data Protection Regulation (GDPR)}
    \subsubsection{Implementation Approaches of the GDPR}
    \subsubsection{International Data Transfer}

    \subsection{Information Security and Computer Crime}
    \subsubsection{Information Security Law}
    \subsubsection{Electronic Signatures and Digital Identities}
    \subsubsection{Cybercrime}
    \lipsum[26][1-3]

    \subsection{Online Media and Telecommunication}
    \subsubsection{Basics of Online Media Law}
    \subsubsection{Social Media and Freedom of Expression}
    \subsubsection{Fundamentals of Telecommunications Law}
    \subsubsection{Internet and Domain Law}

    \section{Literature}
    \begin{itemize}
        \item Lloyd, I. (2020): Information Technology Law. Oxford University Press.
        \item Lutzi, T. (2020): Private International Law Online: Internet Regulation and Civil Liability in the EU. Oxford University Press.
        \item Nirmal, B. C. \& Singh, R. K. (ed.) (2018): Contemporary Issues in International Law. Environment, International Trade, Information Technology and Legal Education. Springer.
        \item Savin, A. (2017): EU Internet Law. Edward Elgar Publishing.
        \item Siems, M. (2018): Comparative law. Cambridge University Press.
        \item Thirlway, H. (2019): The sources of international law. Oxford University Press.
    \end{itemize}

    % ==== Start-up =============================================================================
    \chapter{Start-up}

    \section{Literature}
    \begin{itemize}
        \item Blank, S., Dorf, B. (2020): The Startup Owner's Manual: The Step-By-Step Guide for Building a Great Company. Wiley.
        \item Ries, E. (2011): The Lean Startup: How Today's Entrepreneurs Use Continuous Innovation to Create Radically Successful Businesses. Crown Business.
        \item Maurya, A. (2012): Running Lean: Iterate from Plan A to a Plan That Works. O'Reilly Media.
        \item Osterwalder, A., Pigneur, Y. (2010): Business Model Generation: A Handbook for Visionaries, Game Changers, and Challengers. Wiley.
        \item Kawasaki, G. (2015): The Art of the Start 2.0: The Time-Tested, Battle-Hardened Guide for Anyone Starting Anything. Portfolio.
        \item Bessant, J. \& Tidd, J. (2015). Innovation and Entrepreneurship. 3rd edition, John Wiley \& Sons.
        \item Grant, A. (2016). Originals: How Non-Conformists Move the World. Viking.
        \item Grant, W. (2020). How to Write a Winning Business Plan: A Step-by-Step Guide to Build a Solid Foundation, Attract Investors \& Achieve Success. Walter Grant.
        \item Hoffman, S. (2021). Surviving a Startup: Practical Strategies for Starting a Business, Overcoming Obstacles, and Coming Out on Top. Harper Collins.
        \item Osterwalder, A., Pigneur, Y., Bernarda, G. \& Smith, A. (2010). Value Proposition Design: How to Create Products and Services Customers Want. John Wiley \& Sons.
    \end{itemize}

    % ==== Visualization =============================================================================
    \chapter{Visualization}

    \section{Literature}
    \begin{itemize}
        \item Few, S. (2009). Now you see it: Simple visualization techniques for quantitative analysis. Analytics Press.
        \item Kirk, A. (2016). Data visualization: a successful design process. Packt Publishing Ltd.
        \item McCandless, D. (2010). Information is beautiful. HarperCollins Publishers.
        \item Munzner, T. (2014). Visualization analysis and design. CRC Press.
        \item Yau, N. (2013). Data points: visualization that means something. Wiley.
        \item Few, S. (2013). Information dashboard design: Displaying data for at-a-glance monitoring (2nd ed.). Analytics Press.
        \item Gilliland, M., Tashman, L., \& Sglavo, U. (2016). Business forecasting: Practical problems and solutions. John Wiley \& Sons.
        \item Hyndman, R. (2018). Forecasting: Principles and practice (2nd ed.). OTexts.
        \item Kahneman, D. (2012). Thinking, fast and slow. Penguin Books.
        \item Osterwalder, A., \& Pigneur, Y. (2010). Business model generation. Wiley.
        \item Parmenter, D. (2015). Key performance indicators: Developing, implementing, and using winning KPIs. John Wiley \& Sons.
    \end{itemize}

    % ==== Scientific Writing =============================================================================
    \chapter{Scientific Writing}

    \section{Literature}
    \begin{itemize}
        \item Alley, M. (1996). The craft of scientific writing (3rd ed.). Springer.
        \item Day, R. A., \& Gastel, B. (2012). How to write and publish a scientific paper (7th ed.). Cambridge University Press.
        \item Hofmann, A. H. (2014). Scientific writing and communication: Papers, proposals, and presentations (2nd ed.). Oxford University Press.
        \item Sword, H. (2012). Stylish academic writing. Harvard University Press.
        \item Weissberg, R., \& Buker, S. (1990). Writing up research: Experimental research reports. Prentice Hall.
        \item Turabian, K. L. (2013). A Manual for Writers of Research Papers, Theses, and Dissertations (8th ed.). University of Chicago Press.
        \item Renz, K.-C. (2016). The 1 x 1 of the presentation. For school, study and work. (2nd ed.). Springer Gabler.
    \end{itemize}

    % ==== References =============================================================================
    \chapter{References}
    \printbibliography[heading=none]

\end{multicols}

\end{document}
