% ----------------------------------------------------------------------------------------------------------------------
% Math Learning Cards for AI - Released under the MIT License
% ----------------------------------------------------------------------------------------------------------------------

% User guides
% Latex: https://ftp.agdsn.de/pub/mirrors/latex/dante/macros/latex/base/usrguide.pdf
% Tabularray: https://mirror.physik.tu-berlin.de/pub/CTAN/macros/latex/contrib/tabularray/tabularray.pdf
% TColorbox: https://mirror.clientvps.com/CTAN/macros/latex/contrib/tcolorbox/tcolorbox.pdf
% Posterbox Tutorial: https://mirror.clientvps.com/CTAN/macros/latex/contrib/tcolorbox/tcolorbox-tutorial-poster.pdf#[0,{%22name%22:%22Fit%22}]
% Xcolor: https://mirror.clientvps.com/CTAN/macros/latex/contrib/xcolor/xcolor.pdf
% TikZ: https://tikz.dev/
% Geometry: https://ftp.gwdg.de/pub/ctan/macros/latex/contrib/geometry/geometry.pdf
% fontspec: https://texdoc.org/serve/fontspec/0
% fancyhdr: https://ftp.fau.de/ctan/macros/latex/contrib/fancyhdr/fancyhdr.pdf
% biblatex: https://distrib-coffee.ipsl.jussieu.fr/pub/mirrors/ctan/macros/latex/contrib/biblatex/doc/biblatex.pdf
% amsmath: https://www.latex-project.org/help/documentation/amsldoc.pdf

\documentclass[8pt]{extarticle}


% --- Hyphenation Rules --------------------------------------------------------
\usepackage[USenglish]{babel}

% --- Fonts --------------------------------------------------------------------

% Ensure proper font encoding
\usepackage[T1]{fontenc}

% Load system fonts via fontspec (requires XeLaTeX or LuaLaTeX)
% to get a list of installed fonts: luaotfload-tool --list=format 
\usepackage{fontspec}
\setmainfont{FreeSans}[Scale=1]
\setsansfont{FreeSans}[Scale=1]
\setmonofont{FreeMono}[Scale=1]

% font for the word "Mathematics" in the header
\newfontfamily\mathheaderfont{QTGhoulFace} 


% Emoji support
\usepackage{emoji}

% --- Page layout --------------------------------------------------------------
\usepackage[
    % showframe, 
    nomarginpar, 
    a4paper,
    landscape, 
    right=1.0cm, 
    left=1.0cm, 
    top=1.5cm, 
    bottom=1.5cm,
    headsep=0.25cm,
    headheight=0.75cm,
    footskip=0.5cm,
]{geometry}


\usepackage{fancyhdr}
\pagestyle{fancy}


% --- TColorbox ---------------------------------------------------------------
\usepackage{tcolorbox}
\tcbuselibrary{skins,raster,listings,breakable,minted, poster}


% --- Tables -------------------------------------------------------------------
\usepackage{tabularray}

% --- Color Names -------------------------------------------------------------
\usepackage[dvipsnames]{xcolor}

% --- Graphics -----------------------------------------------------------------
\usepackage{tikz}
\usepackage{graphicx}
\usepackage{svg}
\graphicspath{{./figures/}{./icons/}{./logos/}}

% --- URL and href -----------------------------------------------------------
\colorlet{citecolor}{black}
\colorlet{linkcolor}{black}
\colorlet{urlcolor}{black}
\usepackage[
  bookmarks=true,
  breaklinks=true,
  pdfborder={0 0 0},
  citecolor=citecolor,
  linkcolor=linkcolor,
  urlcolor=urlcolor,
  colorlinks=true,
  linktocpage=false,
  hyperindex=true,
  colorlinks=true,
  linktocpage=false,
  linkbordercolor=white]{hyperref}

% --- Math Environments --------------------------------------------------------
\usepackage{amsmath}
\usepackage{nicematrix}
\NiceMatrixOptions{cell-space-limits = 1pt}
\usepackage{siunitx}

% --- Bibliography -------------------------------------------------------------
\usepackage[autostyle=true]{csquotes} % recommended before biblatex
\usepackage[backend=biber,style=numeric,sorting=nyt]{biblatex}
\addbibresource{../bibliography/references.bib}

\renewcommand{\bibfont}{\normalfont\footnotesize}

% --- Helpers -----------------------------------------------------------------
\usepackage{lipsum}
\usepackage{enumitem}



% --- Default options ---------------------------------------------------------
% \pagestyle{empty}
% \setlength\parindent{0pt}
% \setlength{\tabcolsep}{2pt}
% \baselineskip=0pt
% \setlength\columnsep{1.75mm}
% \setlength{\parskip}{0.1\baselineskip}


\newcommand{\maincolor}{OliveGreen!80!White}
\newcommand{\maincolspan}{4}



% --- TColorbox Skins ----------------------------------------------------------

% Colorbox skin for bash listings
\tcbsubskin{mintedbash}{empty}{
        size=minimal,
        listing engine=minted,
        listing only,
        minted style=friendly,
        minted language=bash,
        minted options={
            fontsize=\footnotesize,
            breaklines,
            autogobble,
        },
        colback=gray!10!white,
        colframe=gray!10!white,
        listing only,
        left=0em,
        enhanced,
}

% Colorbox skin for python listings
\tcbsubskin{mintedpython}{empty}{
        size=minimal,
        listing engine=minted,
        listing only,
        minted style=friendly,
        minted language=python,
        minted options={
            fontsize=\footnotesize,
            breaklines,
            autogobble,
        },
        colback=gray!10!white,
        colframe=gray!10!white,
        listing only,
        left=0em,
        enhanced,
        before skip=0.3\baselineskip,
}

% Skin for header box
\tcbsubskin{headerboxskin}{empty}{
        size=minimal,
        coltitle=black!10!black,
}



% Skin for orange-based section boxes
\tcbsubskin{sectionboxskin}{standard}{
    size=title, 
    left=0.2em,
    right=0.2em,
    arc=0.25mm,
    colback=\maincolor!2!white,
    colframe=\maincolor!75!black,
    coltitle=\maincolor!10!black,
    colbacktitle=\maincolor!75!white,
    fonttitle=\sffamily\bfseries,
    toptitle=0mm,
    bottomtitle=0mm,
    fontupper=\sffamily\small,
    fontlower=\sffamily\small,
    halign=left,
    subtitle style={top=0.4mm, bottom=0mm, boxrule=0.2pt, boxsep=0.1mm,
        colback=\maincolor!50!\maincolor!20!white,
        colupper=\maincolor!50!gray,
        fontupper=\sffamily\bfseries\small,
        coltext=\maincolor!10!black,
        height=1.1\baselineskip,
      }, 
}


% Skin for simple tables inside boxes
\tcbsubskin{boxtablesimple}{empty}{
    size=minimal,
    before skip=0.2\baselineskip,
    after skip=0.2\baselineskip,
    colback=white,
    colframe=gray,
    frame empty,
}





% ----------------------------------------------------------------------------------------------------------------------
% Header and Footer Settings
% ----------------------------------------------------------------------------------------------------------------------

\fancyhead[L]{
    \LARGE {\mathheaderfont Mathematics} \textcolor{\maincolor}{for AI beginners}
}
\fancyhead[C]{}
\fancyhead[R]{}
\fancyfoot[L]{}
\fancyfoot[C]{}
\fancyfoot[R]{\thepage}


% ----------------------------------------------------------------------------------------------------------------------
% Main Document
% ----------------------------------------------------------------------------------------------------------------------

\begin{document}

% === Page 1 =================================================================
\begin{tcbposter}[
        poster = {spacing=0.6em, columns=12},
        boxes = {},
        coverage={left=0pt, right=0pt, top=0pt, bottom=0pt},
    ]

    \newlength{\thispostercolwidth}
    \setlength{\thispostercolwidth}{\tcbpostercolwidth}

    % === XXX ================================================================
    \begin{posterboxenv}[skin=sectionboxskin, adjusted title=Linear Algebra]
        {name=overview,column=1,below=top, span=\maincolspan}

        Linear algebra is the study of vectors and linear functions \textsuperscript{\cite[][p.9]{cherneyLinearAlgebra2013}}.
        \tcbsubtitle{Basics}
        More text

        \tcbsubtitle{N-Dimensional Vector}
        In broad terms, vectors are things one can add and linear functions are functions of vectors that respect vector addition \textsuperscript{\cite[][p.12]{cherneyLinearAlgebra2013}}. Order of components matters. 

        \begin{equation}
            \vec{v} =
                \begin{pNiceMatrix}
                    a_1 \\ a_2 \\ \vdots \\ a_n
                \end{pNiceMatrix}
                \neq
                \begin{pNiceMatrix}
                    \CodeBefore
                    \cellcolor[HTML]{FFFF88}{1-1,2-1}
                    \Body
                    a_2 \\ a_1 \\ \vdots \\ a_n
                \end{pNiceMatrix}
        \end{equation}


        \tcbsubtitle{Vector Calculations}
        \textbf{Length (magnitude)} of vector $\vec{x}$. Also called Euclidean norm or 2-norm \textsuperscript{\cite[][p.90]{cherneyLinearAlgebra2013}}.
 
        \begin{equation}
            \hat{x} = \left\lvert \vec{x} \right\rvert = \sqrt{\sum_{i=1}^n x_i^2} = \sqrt{x_1^2 + x_2^2 + ... + x_n^2} 
        \end{equation}

        \textbf{Normalized vector (unit vector)} in the same direction as $\vec{x}$ \textsuperscript{\cite[][p.262]{cherneyLinearAlgebra2013}}.

        \begin{equation}
            \frac{\vec{x}}{\left\lvert \vec{x} \right\rvert} = 
            \begin{pNiceMatrix}
                    \frac{x_1}{\left\lvert \vec{x} \right\rvert} \\ \frac{x_2}{\left\lvert \vec{x} \right\rvert} \\ \vdots \\ \frac{x_n}{\left\lvert \vec{x} \right\rvert}
            \end{pNiceMatrix}
        \end{equation}

        \textbf{Dot product (scalar product)} of vectors $\vec{v}$ and $\vec{u}$ \textsuperscript{\cite[][p.89]{cherneyLinearAlgebra2013}}. The dot-product of 2 orthogonal vectors is 0 \textsuperscript{\cite[][p.30]{fischerMaschinellesLernenFuer2024}}.

        \begin{equation}
            \vec{v} \cdot \vec{u} =
            \begin{pNiceMatrix}
                v_1 \\ v_2 \\ \vdots \\ v_n
            \end{pNiceMatrix}
            \cdot
            \begin{pNiceMatrix}
                u_1 \\ u_2 \\ \vdots \\ u_n
            \end{pNiceMatrix} = \sum_{i=1}^n v_i u_i = v_1 u_1 + v_2 u_2 + ... + v_n u_n
        \end{equation}

        \textbf{The angle $\Theta$} of vectors $\vec{v}$ and $\vec{u}$ \textsuperscript{\cite[][p.90]{cherneyLinearAlgebra2013}}.

        \begin{equation}
            \vec{u} \cdot \vec{v} = \left\lvert \vec{u} \right\rvert \left\lvert \vec{v} \right\rvert \cos \Theta
            \Rightarrow 
            \cos \Theta = \frac{\vec{u} \cdot \vec{v}}{\left\lvert \vec{u} \right\rvert \left\lvert \vec{v} \right\rvert}
            \Rightarrow 
            \Theta = \arccos \left( \frac{\vec{u} \cdot \vec{v}}{\left\lvert \vec{u} \right\rvert \left\lvert \vec{v} \right\rvert} \right)
        \end{equation}

        \textbf{Cosine similarity} is $\cos \Theta$. In case of normalized vectors $\hat{u}$ and $\hat{v}$, the cosine similarity is their dot product $\cos \Theta = \hat{u} \cdot \hat{v}$. Two vectors are more similar 
        the smaller the angle $\Theta$ between them. Hence, highest similarity $\rightarrow$ lowest similarity $\rightarrow$ highest similarity: $\cos \ang{0} = 1$ (pointing into same direction) $\rightarrow$ $\cos \ang{90} = 0$ (orthogonal vectors) $\rightarrow$ $\cos \ang{180} = -1$ (pointing into opposite directions) \textsuperscript{\cite[][p.31]{fischerMaschinellesLernenFuer2024}}.  


        \tcbsubtitle{Eigenwert}

    \end{posterboxenv}

    % === tbd ================================================================
    \begin{posterboxenv}[skin=sectionboxskin, adjusted title=Overview]
        {name=overview,column=5,below=top, span=4}
        Rust is a general-purpose programming language focused on performance, memory safety without garbage collector, and concurrency, using an ownership system preventing memory errors.

        \tcbsubtitle{\href{https://rustup.rs/}{Installation}}
        More text

    \end{posterboxenv}

\end{tcbposter}

\pagebreak

% === Page 2 =================================================================

% === Page 1 =================================================================
\begin{tcbposter}[
        poster = {spacing=0.6em, columns=12},
        boxes = {},
        coverage={left=0pt, right=0pt, top=0pt, bottom=0pt},
    ]

    % === References ================================================================
    \begin{posterboxenv}[skin=sectionboxskin, adjusted title=References]
        {name=overview,column=9,below=top, span=4}
        \printbibliography[heading=none]

    \end{posterboxenv}

\end{tcbposter}

% \printbibheading
% % \printbibliography[type=book,heading=subbibliography,title={Book Sources}]
% \printbibliography[nottype=book,heading=subbibliography,title={Other Sources}]

\end{document}
