% ----------------------------------------------------------------------------------------------------------------------
% AI & Data Analytics Learning Cards - Released under the MIT License
% ----------------------------------------------------------------------------------------------------------------------

% User guides
% Latex: https://ftp.agdsn.de/pub/mirrors/latex/dante/macros/latex/base/usrguide.pdf
% Tabularray: https://mirror.physik.tu-berlin.de/pub/CTAN/macros/latex/contrib/tabularray/tabularray.pdf
% TColorbox: https://mirror.clientvps.com/CTAN/macros/latex/contrib/tcolorbox/tcolorbox.pdf
% Posterbox Tutorial: https://mirror.clientvps.com/CTAN/macros/latex/contrib/tcolorbox/tcolorbox-tutorial-poster.pdf#[0,{%22name%22:%22Fit%22}]
% Xcolor: https://mirror.clientvps.com/CTAN/macros/latex/contrib/xcolor/xcolor.pdf
% TikZ and PGFPlots: https://tikz.dev/ and https://tikz.dev/pgfplots/
%                    https://tug.ctan.org/info/visualtikz/VisualTikZ.pdf 
% SVG Package: https://ctan.org/pkg/svg?lang=en
% Geometry: https://ftp.gwdg.de/pub/ctan/macros/latex/contrib/geometry/geometry.pdf
% fontspec: https://texdoc.org/serve/fontspec/0
% fancyhdr: https://ftp.fau.de/ctan/macros/latex/contrib/fancyhdr/fancyhdr.pdf
% biblatex: https://distrib-coffee.ipsl.jussieu.fr/pub/mirrors/ctan/macros/latex/contrib/biblatex/doc/biblatex.pdf
% amsmath: https://www.latex-project.org/help/documentation/amsldoc.pdf
% grapheur: https://ctan.tetaneutral.net/graphics/pgf/contrib/tkz-grapheur/doc/tkz-grapheur-doc-en.pdf
% enumitem: https://mirror.ibcp.fr/pub/CTAN/macros/latex/contrib/enumitem/enumitem.pdf

\documentclass[8pt]{extarticle}


% --- Hyphenation Rules --------------------------------------------------------
\usepackage[USenglish]{babel}

% --- Fonts --------------------------------------------------------------------

% Ensure proper font encoding
\usepackage[T1]{fontenc}

% Load system fonts via fontspec (requires XeLaTeX or LuaLaTeX)
% to get a list of installed fonts: luaotfload-tool --list=format 
\usepackage{fontspec}
\setmainfont{FreeSans}[Scale=1]
\setsansfont{FreeSans}[Scale=1]
\setmonofont{FreeMono}[Scale=1]

% font for the word "Mathematics" in the header
\newfontfamily\headerfont{QTGhoulFace} 


% Emoji support
\usepackage{emoji}

% --- Page layout --------------------------------------------------------------
\usepackage[
    % showframe, 
    nomarginpar, 
    a4paper,
    portrait, 
    right=0.75cm, 
    left=0.75cm, 
    top=1.25cm, 
    bottom=1cm,
    headsep=0.25cm,
    headheight=0.75cm,
    footskip=0.5cm,
]{geometry}


\usepackage{fancyhdr}
\pagestyle{fancy}


% --- TColorbox ---------------------------------------------------------------
\usepackage{tcolorbox}
\tcbuselibrary{skins,raster,listings,breakable,minted, poster}


% --- Tables -------------------------------------------------------------------
\usepackage{tabularray}

% --- Color Names -------------------------------------------------------------
\usepackage[dvipsnames]{xcolor}

% --- Graphics -----------------------------------------------------------------
\usepackage{graphicx}
\usepackage{svg}
\graphicspath{{./figures/}{./icons/}{./logos/}}
\usepackage{tikz}
\usepackage{pgfplots}
\pgfplotsset{compat=1.18}
\usetikzlibrary{shapes.geometric}    % for shapes
\usetikzlibrary{arrows.meta}         % for arrows, Arrow tip library 
\usetikzlibrary{positioning}         % for relative positioning
\usetikzlibrary{calc}                % for calculations to make complex coordinate calculations
%\usetikzlibrary{backgrounds}        % Background Library "defines background for pictures". To use this in a Tikzpicture, an option is passed, e.g. \begin{tikzpicture}[show background rectangle], with a background rectangle style defined before the picture. (e.g. \tikzset{background rectangle/.style={<define background rectangle style here>}}
% \usetikzlibrary{calendars}         % for calendar drawings
% \usetikzlibrary{fadings}           % for fading effects
% \usetikzlibrary{patterns}          % for patterns
% \usetikzlibrary{shadows}           % for shadow effects 
\usetikzlibrary{chains}            % for chain effects
\usetikzlibrary{fit}               % for fitting nodes together
% \usetikzlibrary{er}                % for entity relationship diagrams
% \usetikzlibrary{intersections}     % for calculating intersections of paths
% \usetikzlibrary{mindmap}           % for mind maps
\usetikzlibrary{matrix}            % for matrix drawings
\usetikzlibrary{angles}           % for angle markings
\usetikzlibrary{quotes}           % for quotes in angles
% \usetikzlibrary{trees}             % for tree drawings
% \usetikzlibrary{decorations.pathmorphing}  % for path decorations like zigzag lines
\usetikzlibrary{decorations.pathreplacing} % for braces and other path decorations
% \usetikzlibrary{decorations.markings}      % for markings along a path
% \usetikzlibrary{decorations.shapes}        % for shape decorations
% \usetikzlibrary{decorations.text}          % for text decorations along a path
% \usetikzlibrary{pgfplots.colorbrewer}      % for color maps

% \usepackage{forest}               % for drawing trees https://ctan.ceremade.dauphine.fr/graphics/pgf/contrib/forest/forest-doc.pdf

\usepackage{adjustbox}          % for adjusting boxes

\usepackage[labelfont=sf]{caption}  % for figure captions on non-floats

\newcommand{\flowchartstylesmaincolor}{OliveGreen!80!White}
\newcommand{\flowchartstyleslinewidth}{1.25pt}

\tikzstyle{diam} = [diamond, aspect=2, draw, fill=red!40, text width=6em,text centered, line width=\flowchartstyleslinewidth ]
\tikzstyle{block} = [rectangle, draw=\flowchartstylesmaincolor, text width=3cm, text centered, rounded corners, minimum height=2em, line width=\flowchartstyleslinewidth ]
\tikzstyle{round} = [circle, draw=\flowchartstylesmaincolor, text centered, rounded corners, minimum height=2em, line width=\flowchartstyleslinewidth ]
\tikzstyle{trap} = [trapezium, trapezium left angle=70, trapezium right angle=110, minimum height=1em, text centered, draw=\flowchartstylesmaincolor!50!red, fill=\flowchartstylesmaincolor!20!green!30, line width=\flowchartstyleslinewidth ]
\tikzstyle{rect} = [rectangle, minimum width=3cm, minimum height=1cm, text centered, draw=\flowchartstylesmaincolor, line width=\flowchartstyleslinewidth, align=center, inner sep=1em ]
\tikzstyle{line} = [draw=\flowchartstylesmaincolor, -latex, line width=\flowchartstyleslinewidth ]
\tikzstyle{circ} = [round, draw=\flowchartstylesmaincolor,minimum width=2cm, align=center, line width=\flowchartstyleslinewidth ]


  % load flowchart styles


% --- URL and href -----------------------------------------------------------
\colorlet{citecolor}{black}
\colorlet{linkcolor}{black}
\colorlet{urlcolor}{black}
\usepackage[
  bookmarks=true,
  breaklinks=true,
  pdfborder={0 0 0},
  citecolor=citecolor,
  linkcolor=linkcolor,
  urlcolor=urlcolor,
  colorlinks=true,
  linktocpage=false,
  hyperindex=true,
  colorlinks=true,
  linktocpage=false,
  linkbordercolor=white]{hyperref}

% --- Math Environments --------------------------------------------------------
\usepackage{amsmath}
\usepackage{nicematrix}
\NiceMatrixOptions{cell-space-limits = 1pt}
\usepackage{siunitx}

% --- Bibliography -------------------------------------------------------------
\usepackage[autostyle=true]{csquotes} % recommended before biblatex
\usepackage[backend=biber,style=numeric,sorting=nyt]{biblatex}
\addbibresource{../bibliography/references.bib}

\renewcommand{\bibfont}{\normalfont\footnotesize}


% --- Enumerations -----------------------------------------------------------------
\usepackage{enumitem}

\setlist{noitemsep}
\setlist[1]{labelindent=\parindent} % < Usually a good idea
\setlist[itemize]{leftmargin=*}
\setlist[itemize,1]{label=-}

% --- Helpers -----------------------------------------------------------------
\usepackage{lipsum}
\usepackage{multicol}


% --- Options ---------------------------------------------------------
% \pagestyle{empty}
% \setlength\parindent{0pt}
% \setlength{\tabcolsep}{2pt}
% \baselineskip=0pt
\setlength{\columnsep}{0.75cm}
% \setlength{\parskip}{0.1\baselineskip}

\setlist[enumerate]{itemsep=0.2em, topsep=0.2em, parsep=0em, partopsep=0em}

\captionsetup{font=small,labelfont={bf,sf}}
% \captionsetup[sub]{font=small,labelfont={bf,sf}}


\newcommand{\maincolor}{LimeGreen!90!Gray}
\newcommand{\maincolspan}{4}
\newcommand{\graphscale}{0.9}

\title{Advanced Mathematics and Statistics for AI Beginners}
\author{Bernhard Gerlach}
\date{\today}

% --- TColorbox Skins ----------------------------------------------------------

% Colorbox skin for bash listings
\tcbsubskin{mintedbash}{empty}{
        size=minimal,
        listing engine=minted,
        listing only,
        minted style=friendly,
        minted language=bash,
        minted options={
            fontsize=\footnotesize,
            breaklines,
            autogobble,
        },
        colback=gray!10!white,
        colframe=gray!10!white,
        listing only,
        left=0em,
        enhanced,
}

% Colorbox skin for python listings
\tcbsubskin{mintedpython}{empty}{
        size=minimal,
        listing engine=minted,
        listing only,
        minted style=friendly,
        minted language=python,
        minted options={
            fontsize=\footnotesize,
            breaklines,
            autogobble,
        },
        colback=gray!10!white,
        colframe=gray!10!white,
        listing only,
        left=0em,
        enhanced,
        before skip=0.3\baselineskip,
}

% Skin for header box
\tcbsubskin{headerboxskin}{empty}{
        size=minimal,
        coltitle=black!10!black,
}



% Skin for orange-based section boxes
\tcbsubskin{sectionboxskin}{standard}{
    size=title, 
    left=0.2em,
    right=0.2em,
    arc=0.25mm,
    colback=\maincolor!2!white,
    colframe=\maincolor!75!black,
    coltitle=\maincolor!10!black,
    colbacktitle=\maincolor!75!white,
    fonttitle=\sffamily\bfseries,
    toptitle=0mm,
    bottomtitle=0mm,
    fontupper=\sffamily\small,
    fontlower=\sffamily\small,
    lower separated=false,
    valign lower=center,
    halign=left,
    subtitle style={top=0.4mm, bottom=0mm, boxrule=0.2pt, boxsep=0.1mm,
        colback=\maincolor!50!\maincolor!20!white,
        colupper=\maincolor!50!gray,
        fontupper=\sffamily\bfseries\small,
        coltext=\maincolor!10!black,
        height=1.1\baselineskip,
      }, 
}


% Skin for sub-section boxes
\tcbsubskin{subsectionboxskin}{standard}{
    size=fbox, 
    top=0.4em,
    % left=0.2em,
    % right=0.2em,
    % arc=0.25mm,
    % top=0.4mm, 
    % bottom=0mm, 
    % boxrule=0.2pt, 
    % boxsep=0.1mm,
    frame empty,
    % colback=\maincolor!2!white,
    colback=white,
    colbacktitle=\maincolor!50!\maincolor!20!white,
    fonttitle=\sffamily\bfseries\small,
    coltitle=\maincolor!10!black,
    % height=1.1\baselineskip,    
}


% Skin for simple tables inside boxes
\tcbsubskin{boxtablesimple}{empty}{
    size=minimal,
    before skip=0.2\baselineskip,
    after skip=0.2\baselineskip,
    colback=white,
    colframe=gray,
    frame empty,
}


\tcbsubskin{sectionraster}{sectionboxskin}{
    raster columns=6,
    raster equal height=rows,
    raster row skip=0.25cm,
    raster column skip=0.25cm,
    skin=sectionboxskin,
}





% ----------------------------------------------------------------------------------------------------------------------
% Header and Footer Settings
% ----------------------------------------------------------------------------------------------------------------------

\fancyhead[L]{
    \LARGE \textcolor{\maincolor}{\headerfont Mathematics} for AI beginners
}
\fancyhead[C]{}
\fancyhead[R]{}
\fancyfoot[L]{}
\fancyfoot[C]{}
\fancyfoot[R]{\thepage}


% ----------------------------------------------------------------------------------------------------------------------
% Main Document
% ----------------------------------------------------------------------------------------------------------------------

\begin{document}

\begin{multicols}{2}[
        \section{Advanced Mathematics}
    ]
    \subsection{Calculus}
    \subsubsection{Differentiation}
    \begin{tcbitemize}[ skin=sectionraster ]
        \tcbitem[title=Gradient, raster multicolumn=6]
        The gradient for any function $y = f(x)$ is defined as follows. The gradient is also called the first derivative. The three notations $f'(x)$, $dy/dx$ and $df(x)/dx$ are interchangeable\textsuperscript{\cite[][p89]{yangMathematicsCivilEngineers2018} and \cite[][p393]{bronsteinTaschenbuchMathematik2001}}.
        \tcblower
        \begin{equation}
            f'(x) = \frac{dy}{dx} = \frac{df(x)}{dx} = \lim_{\Delta x \to 0} \frac{f(x + \Delta x) - f(x)}{\Delta x}
        \end{equation}

        \vspace*{4ex}
        \centering
        \captionsetup{hypcap=false}
        \begin{tikzpicture}
            \begin{axis}[
                width=8cm,
                height=6cm,
                axis lines=middle,
                axis line style={thick, -Stealth},
                xlabel={$x$},
                ylabel={$y = f(x)$},
                xmin=0, xmax=3,
                ymin=-0.5, ymax=5,
                samples=100,
                domain=0:2.2,
                legend style={font=\footnotesize, at={(1.1,1.1)}, anchor=north east},
                clip=false,
                ylabel style={at={(ticklabel cs:1.0)},anchor=south}, 
                xlabel style={at={(1.0,0.1)},anchor=south}, 
            ]
            % Plot y = x^2
            \addplot[thick, \maincolor!80!black, smooth] {x^2};
            \addlegendentry{$y = x^2$}
            
            % Tangent at point P (x=1, y=1)
            % Derivative: y' = 2x, at x=1: y'(1) = 2
            % Tangent line: y - 1 = 2(x - 1) => y = 2x - 1
            \addplot[thick, red, dashed, domain=0.5:2.5] {2*x - 1};
            \addlegendentry{Tangent at $P$}
            
            % Mark point P
            \addplot[only marks, mark=*, mark size=2pt, blue] coordinates {(1,1)};
            \node[above left, font=\footnotesize] at (axis cs:1,1) {$P$};
            
            % Line M: vertical line from P down to x-axis (at x=1)
            \draw[thick, blue] (axis cs:1,1) -- (axis cs:1,0);
            \node[right, font=\footnotesize, blue] at (axis cs:1,0.5) {$y$};
            
            % Line N: horizontal line from P to x=2 (parallel to x-axis)
            \draw[thick, blue] (axis cs:1,1) -- (axis cs:2,1);
            
            % Vertical line at x=2 from x-axis up
            % y=x^2 at x=2: y=4, tangent y=2x-1 at x=2: y=3
            \draw[thick, blue] (axis cs:2,0) -- (axis cs:2,4);
            
            % Mark intersection points
            \addplot[only marks, mark=*, mark size=1.5pt, blue] coordinates {(2,1)};  % N meets vertical at x=2
            \addplot[only marks, mark=*, mark size=1.5pt, blue] coordinates {(2,3)};  % tangent at x=2
            \addplot[only marks, mark=*, mark size=1.5pt, blue] coordinates {(2,4)};  % curve at x=2
            
            % Label dy: from N (y=1) to tangent (y=3) at x=2
            \draw[thick, orange, decorate, decoration={brace, amplitude=4pt, mirror}] (axis cs:2.1,1) -- (axis cs:2.1,3);
            \node[right, font=\footnotesize, orange] at (axis cs:2.15,2) {$dy$};
            
            % Label delta y: from N (y=1) to curve (y=4) at x=2
            \draw[thick, purple, decorate, decoration={brace, amplitude=4pt, mirror}] (axis cs:2.4,1) -- (axis cs:2.4,4);
            \node[right, font=\footnotesize, purple] at (axis cs:2.45,2.5) {$\Delta y$};
            
            % Label x: from origin to where M hits x-axis (x=1)
            \draw[thick, gray, decorate, decoration={brace, amplitude=4pt, mirror}] (axis cs:0,-0.45) -- (axis cs:1,-0.45);
            \node[below, font=\footnotesize, gray] at (axis cs:0.5,-0.6) {$x$};
            
            % Label delta x = dx: from M (x=1) to x=2
            \draw[thick, gray, decorate, decoration={brace, amplitude=4pt, mirror}] (axis cs:1,-0.45) -- (axis cs:2,-0.45);
            \node[below, font=\footnotesize, gray] at (axis cs:1.5,-0.6) {$\Delta x = dx$};
            
            % Angle alpha where tangent crosses x-axis
            % Tangent y=2x-1 crosses x-axis at x=0.5
            % where A, B, and C are the vertices, and B is the apex
            \path (1,0) coordinate (A) -- (0.5,0) coordinate (B) -- (1,1) coordinate (C) pic ["$\alpha$", draw, -latex, angle radius=25pt] {angle};
            % \draw[thick, black] (axis cs:0.8,0) arc[start angle=0, end angle=63.43, radius=0.15cm];
            % \node[above right, font=\footnotesize] at (axis cs:0.7,0.05) {$\alpha$};
            
            \end{axis}
        \end{tikzpicture}
        \captionof{figure}{Graph of example function $y = x^2$ with tangent line at point $P(1,1)$. The tangent's gradient is defined by angle $\alpha$, known as the angle of inclination of the tangent line.}\label{fig:derivative}

        
    \end{tcbitemize}    
    \begin{enumerate}[label*=\arabic*.]
        \item ...
        \item Integration
        \item Partial Differentiation
        \item Vector Analysis
    \end{enumerate}

    \subsection{Integral Transformations}
    \begin{enumerate}[label*=\arabic*.]
        \item Convolution
        \item Complex Numbers
        \item Fourier Series
        \item Fourier Transformation
    \end{enumerate}

    \subsection{Vector Algebra}
    \subsubsection{Scalars and Vectors}

    \begin{tcbitemize}[ skin=sectionraster ]
        \tcbitem[title=N-Dimensional Vector, raster multicolumn=6]
        In broad terms, vectors are things one can add and linear functions are functions of vectors that respect vector addition\textsuperscript{\cite[][p.12]{cherneyLinearAlgebra2013}}. Order of components matters.

        \begin{equation}
            \vec{v} =
            \begin{pNiceMatrix}
                a_1 \\ a_2 \\ \vdots \\ a_n
            \end{pNiceMatrix}
            \neq
            \begin{pNiceMatrix}
                \CodeBefore
                \cellcolor[HTML]{FFFF88}{1-1,2-1}
                \Body
                a_2 \\ a_1 \\ \vdots \\ a_n
            \end{pNiceMatrix}
        \end{equation}

        \tcbitem[title=Length (Magnitude), raster multicolumn=6]
        Length (magnitude) of vector $\vec{x}$. Also called Euclidean norm or 2-norm \textsuperscript{\cite[][p.90]{cherneyLinearAlgebra2013}}.
        \tcblower
        \begin{equation}
            \hat{x} = \left\lvert \vec{x} \right\rvert = \sqrt{\sum_{i=1}^n x_i^2} = \sqrt{x_1^2 + x_2^2 + ... + x_n^2}
        \end{equation}

        \tcbitem[title=Normalized vector (unit vector), raster multicolumn=6]
        Normalized vector (unit vector) in the same direction as $\vec{x}$ \textsuperscript{\cite[][p.262]{cherneyLinearAlgebra2013}}.
        \tcblower
        \begin{equation}
            \frac{\vec{x}}{\left\lvert \vec{x} \right\rvert} =
            \begin{pNiceMatrix}
                \frac{x_1}{\left\lvert \vec{x} \right\rvert} \\ \frac{x_2}{\left\lvert \vec{x} \right\rvert} \\ \vdots \\ \frac{x_n}{\left\lvert \vec{x} \right\rvert}
            \end{pNiceMatrix}
        \end{equation}

    \end{tcbitemize}

    \subsubsection{Addition and Subtraction of Vectors}
    \subsubsection{Multiplication of Vectors, Vector Product, Scalar Product}
    \begin{tcbitemize}[ skin=sectionraster ]

        \tcbitem[title=Scalar Product (dot product), raster multicolumn=6]
        Scalar product (dot product) of vectors $\vec{v}$ and $\vec{u}$ \textsuperscript{\cite[][p.89]{cherneyLinearAlgebra2013}}. The dot-product of 2 orthogonal vectors is 0 \textsuperscript{\cite[][p.30]{fischerMaschinellesLernenFuer2024}}.
        \tcblower
        \begin{equation}
            \vec{v} \cdot \vec{u} =
            \begin{pNiceMatrix}
                v_1 \\ v_2 \\ \vdots \\ v_n
            \end{pNiceMatrix}
            \cdot
            \begin{pNiceMatrix}
                u_1 \\ u_2 \\ \vdots \\ u_n
            \end{pNiceMatrix} = \sum_{i=1}^n v_i u_i = v_1 u_1 + v_2 u_2 + ... + v_n u_n
        \end{equation}

        \tcbitem[title=Angle and Cosine Similarity, raster multicolumn=6]
        The angle $\Theta$ of vectors $\vec{v}$ and $\vec{u}$ \textsuperscript{\cite[][p.90]{cherneyLinearAlgebra2013}}.

        \begin{equation}
            \vec{u} \cdot \vec{v} = \left\lvert \vec{u} \right\rvert \left\lvert \vec{v} \right\rvert \cos \Theta
            \Rightarrow
            \cos \Theta = \frac{\vec{u} \cdot \vec{v}}{\left\lvert \vec{u} \right\rvert \left\lvert \vec{v} \right\rvert}
            \Rightarrow
            \Theta = \arccos \left( \frac{\vec{u} \cdot \vec{v}}{\left\lvert \vec{u} \right\rvert \left\lvert \vec{v} \right\rvert} \right)
        \end{equation}

        \vspace{2ex}
        
        Cosine similarity is $\cos \Theta$. In case of normalized vectors $\hat{u}$ and $\hat{v}$, the cosine similarity is their dot product $\cos \Theta = \hat{u} \cdot \hat{v}$. Two vectors are more similar
        the smaller the angle $\Theta$ between them. Hence, highest similarity $\rightarrow$ lowest similarity $\rightarrow$ highest similarity: $\cos \ang{0} = 1$ (pointing into same direction) $\rightarrow$ $\cos \ang{90} = 0$ (orthogonal vectors) $\rightarrow$ $\cos \ang{180} = -1$ (pointing into opposite directions) \textsuperscript{\cite[][p.31]{fischerMaschinellesLernenFuer2024}}.


    \end{tcbitemize}

    \subsection{Vector Calculus}
    \begin{enumerate}[label*=\arabic*.]
        \item Differentiation of Vectors
        \item Integration of Vectors
        \item Scalar and Vector Fields
        \item Vector Operators
    \end{enumerate}

    \subsection{Matrices and Vector Spaces}
    \begin{enumerate}[label*=\arabic*.]
        \item Basic Matrix Algebra and Systems of Linear Equations
        \item Transpose, Trace, Determinant, and Inverse of a Matrix
        \item Eigenvalues, Eigenvectors, and Diagonalization
        \item Tensors
    \end{enumerate}

    \subsection{Information Theory}
    \begin{enumerate}[label*=\arabic*.]
        \item Mean Squared Error (MSE) and Simple Linear Regression
        \item Area Under the ROC Curve and Gini Index
        \item Entropy
        \item Cross Entropy
    \end{enumerate}


\end{multicols}

\pagebreak

\begin{multicols}{2}
    \section{Advanced Statistics}
    \subsection{Introduction to Statistics}
    \begin{enumerate}[label*=\arabic*.]
        \item Random Variables
        \item Kolmogorov Axioms
        \item Probability Distributions
        \item Decomposing probability distributions
        \item Expectation Values and Moments
        \item Central Limit Theorem
        \item Sufficient Statistics
        \item Problems of Dimensionality
        \item Component Analysis and Discriminants
    \end{enumerate}
    \subsection{Important Probability Distributions and their Applications}
    \begin{enumerate}[label*=\arabic*.]
        \item Binomial Distribution
        \item Gauss or Normal Distribution
        \item Poisson and Gamma-Poisson Distribution
        \item Weibull Distribution
    \end{enumerate}
    \subsection{Bayesian Statistics}
    \begin{enumerate}[label*=\arabic*.]
        \item Bayes’ Rule
        \item Estimating the Prior, Benford’s Law, Jeffry’s Rule
        \item Conjugate Prior
        \item Bayesian \& Frequentist Approach
    \end{enumerate}
    \subsection{Descriptive Statistics}
    \begin{enumerate}[label*=\arabic*.]
        \item Mean, Median, Mode, Quantiles
        \item Variance, Skewness, Kurtosis
    \end{enumerate}
    \subsection{Data Visualization}
    \begin{enumerate}[label*=\arabic*.]
        \item General Principles of Dataviz/Visual Communication
        \item 1D, 2D Histograms
        \item Box Plot, Violin Plot
        \item Scatter Plot, Scatter Plot Matrix, Profile Plot
        \item Bar Chart
    \end{enumerate}
    \subsection{Parameter Estimation}
    \begin{enumerate}[label*=\arabic*.]
        \item Maximum Likelihood
        \item Ordinary Least Squares
        \item Expectation Maximization (EM)
        \item Lasso and Ridge Regularization
        \item Propagation of Uncertainties
    \end{enumerate}
    \subsection{Hypothesis Test}
    \begin{enumerate}[label*=\arabic*.]
        \item Error of 1st and 2nd Kind
        \item Multiple Hypothesis Tests
        \item p-Value
    \end{enumerate}


\end{multicols}

\section{References}
\printbibliography[heading=none]

\pagebreak

\section{Appendix}
\subsection{Course Content Overview}
\subsubsection{Advanced Mathematics}
\begin{enumerate}[label=\arabic*.]
    \item Calculus
          \begin{enumerate}[label*=\arabic*.]
              \item Differentiation
              \item Integration
              \item Partial Differentiation
              \item Vector Analysis
          \end{enumerate}
    \item Integral Transformations
          \begin{enumerate}[label*=\arabic*.]
              \item Convolution
              \item Complex Numbers
              \item Fourier Series
              \item Fourier Transformation
          \end{enumerate}
    \item Vector Algebra
          \begin{enumerate}[label*=\arabic*.]
              \item Scalars and Vectors
              \item Addition and Subtraction of Vectors
              \item Multiplication of Vectors, Vector Product, Scalar Product
          \end{enumerate}
    \item Vector Calculus
          \begin{enumerate}[label*=\arabic*.]
              \item Differentiation of Vectors
              \item Integration of Vectors
              \item Scalar and Vector Fields
              \item Vector Operators
          \end{enumerate}
    \item Matrices and Vector Spaces
          \begin{enumerate}[label*=\arabic*.]
              \item Basic Matrix Algebra and Systems of Linear Equations
              \item Transpose, Trace, Determinant, and Inverse of a Matrix
              \item Eigenvalues, Eigenvectors, and Diagonalization
              \item Tensors
          \end{enumerate}
    \item Information Theory
          \begin{enumerate}[label*=\arabic*.]
              \item Mean Squared Error (MSE) and Simple Linear Regression
              \item Area Under the ROC Curve and Gini Index
              \item Entropy
              \item Cross Entropy
          \end{enumerate}
\end{enumerate}
\subsubsection{Advanced Statistics}
\begin{enumerate}
    \item Introduction to Statistics
          \begin{enumerate}[label*=\arabic*.]
              \item Random Variables
              \item Kolmogorov Axioms
              \item Probability Distributions
              \item Decomposing probability distributions
              \item Expectation Values and Moments
              \item Central Limit Theorem
              \item Sufficient Statistics
              \item Problems of Dimensionality
              \item Component Analysis and Discriminants
          \end{enumerate}
    \item Important Probability Distributions and their Applications
          \begin{enumerate}[label*=\arabic*.]
              \item Binomial Distribution
              \item Gauss or Normal Distribution
              \item Poisson and Gamma-Poisson Distribution
              \item Weibull Distribution
          \end{enumerate}
    \item Bayesian Statistics
          \begin{enumerate}[label*=\arabic*.]
              \item Bayes’ Rule
              \item Estimating the Prior, Benford’s Law, Jeffry’s Rule
              \item Conjugate Prior
              \item Bayesian \& Frequentist Approach
          \end{enumerate}
    \item Descriptive Statistics
          \begin{enumerate}[label*=\arabic*.]
              \item Mean, Median, Mode, Quantiles
              \item Variance, Skewness, Kurtosis
          \end{enumerate}
    \item Data Visualization
          \begin{enumerate}[label*=\arabic*.]
              \item General Principles of Dataviz/Visual Communication
              \item 1D, 2D Histograms
              \item Box Plot, Violin Plot
              \item Scatter Plot, Scatter Plot Matrix, Profile Plot
              \item Bar Chart
          \end{enumerate}
    \item Parameter Estimation
          \begin{enumerate}[label*=\arabic*.]
              \item Maximum Likelihood
              \item Ordinary Least Squares
              \item Expectation Maximization (EM)
              \item Lasso and Ridge Regularization
              \item Propagation of Uncertainties
          \end{enumerate}
    \item Hypothesis Test
          \begin{enumerate}[label*=\arabic*.]
              \item Error of 1st and 2nd Kind
              \item Multiple Hypothesis Tests
              \item p-Value
          \end{enumerate}
\end{enumerate}
\end{document}

%     % === Advanced Mathematics ================================================================
%     % Reduce vertical space between items in all enumerations
%     \setlist[enumerate]{itemsep=0.2em, topsep=0.2em, parsep=0em, partopsep=0em}

%     \begin{posterboxenv}[skin=sectionboxskin, adjusted title=Advanced Mathematics]
%         {name=overview,column=1,below=top, span=4}
%         \tcbsubtitle{Content}
%             \begin{enumerate}[label=\arabic*.]
%                 \item Calculus
%                 \begin{enumerate}[label*=\arabic*.]
%                     \item Differentiation
%                     \item Integration
%                     \item Partial Differentiation
%                     \item Vector Analysis
%                 \end{enumerate}
%                 \item Integral Transformations
%                 \begin{enumerate}[label*=\arabic*.]
%                     \item Convolution
%                     \item Complex Numbers
%                     \item Fourier Series
%                     \item Fourier Transformation
%                 \end{enumerate}
%                 \item Vector Algebra
%                 \begin{enumerate}[label*=\arabic*.]
%                     \item Scalars and Vectors
%                     \item Addition and Subtraction of Vectors
%                     \item Multiplication of Vectors, Vector Product, Scalar Product
%                 \end{enumerate}
%                 \item Vector Calculus
%                 \begin{enumerate}[label*=\arabic*.]
%                     \item Differentiation of Vectors
%                     \item Integration of Vectors
%                     \item Scalar and Vector Fields
%                     \item Vector Operators
%                 \end{enumerate}
%                 \item Matrices and Vector Spaces
%                 \begin{enumerate}[label*=\arabic*.]
%                     \item Basic Matrix Algebra and Systems of Linear Equations
%                     \item Transpose, Trace, Determinant, and Inverse of a Matrix
%                     \item Eigenvalues, Eigenvectors, and Diagonalization
%                     \item Tensors
%                 \end{enumerate}
%                 \item Information Theory
%                 \begin{enumerate}[label*=\arabic*.]
%                     \item Mean Squared Error (MSE) and Simple Linear Regression
%                     \item Area Under the ROC Curve and Gini Index
%                     \item Entropy
%                     \item Cross Entropy
%                 \end{enumerate}
%             \end{enumerate}
%         \tcbsubtitle{Books}
%             \begin{enumerate}
%                 \item Mathai, A. M., \& Haubold, H. J. (2017). Linear algebra, a course for physicists and engineers (1sted.) De Gruyter.
%                 \item Riley, K. F., Hobson, M. P, \& Bence, S. J. (2006). Mathematical methods for physics and engineering (2nd ed.). Cambridge University Press.
%                 \item Yang, X.-S. (2018). Mathematics for Civil Engineers: An Introduction. Dunedin Academic Press.
%             \end{enumerate}
%     \end{posterboxenv}

%     % === Advanced Statistics ================================================================
%     \begin{posterboxenv}[skin=sectionboxskin, adjusted title=Advanced Statistics]
%         {name=overview,column=5,below=top, span=4}
%         \tcbsubtitle{Content}
%             \begin{enumerate}
%                 \item Introduction to Statistics
%                 \begin{enumerate}[label*=\arabic*.]
%                     \item Random Variables
%                     \item Kolmogorov Axioms
%                     \item Probability Distributions
%                     \item Decomposing probability distributions
%                     \item Expectation Values and Moments
%                     \item Central Limit Theorem
%                     \item Sufficient Statistics
%                     \item Problems of Dimensionality
%                     \item Component Analysis and Discriminants
%                 \end{enumerate}
%                 \item Important Probability Distributions and their Applications
%                 \begin{enumerate}[label*=\arabic*.]
%                     \item Binomial Distribution
%                     \item Gauss or Normal Distribution
%                     \item Poisson and Gamma-Poisson Distribution
%                     \item Weibull Distribution
%                 \end{enumerate}
%                 \item Bayesian Statistics
%                 \begin{enumerate}[label*=\arabic*.]
%                     \item Bayes’ Rule
%                     \item Estimating the Prior, Benford’s Law, Jeffry’s Rule
%                     \item Conjugate Prior
%                     \item Bayesian \& Frequentist Approach
%                 \end{enumerate}
%                 \item Descriptive Statistics
%                 \begin{enumerate}[label*=\arabic*.]
%                     \item Mean, Median, Mode, Quantiles
%                     \item Variance, Skewness, Kurtosis
%                 \end{enumerate}
%                 \item Data Visualization
%                 \begin{enumerate}[label*=\arabic*.]
%                     \item General Principles of Dataviz/Visual Communication
%                     \item 1D, 2D Histograms
%                     \item Box Plot, Violin Plot
%                     \item Scatter Plot, Scatter Plot Matrix, Profile Plot
%                     \item Bar Chart
%                 \end{enumerate}
%                 \item Parameter Estimation
%                 \begin{enumerate}[label*=\arabic*.]
%                     \item Maximum Likelihood
%                     \item Ordinary Least Squares
%                     \item Expectation Maximization (EM)
%                     \item Lasso and Ridge Regularization
%                     \item Propagation of Uncertainties
%                 \end{enumerate}
%                 \item Hypothesis Test
%                 \begin{enumerate}[label*=\arabic*.]
%                     \item Error of 1st and 2nd Kind
%                     \item Multiple Hypothesis Tests
%                     \item p-Value
%                 \end{enumerate}
%             \end{enumerate}
%         \tcbsubtitle{Books}
%             \begin{enumerate}
%                 \item Bruce, P., \& Bruce, A. (2017). Statistics for data scientists: 50 essential concepts. O’Reilley Publishing.
%                 \item Downey, A. (2013). Think Bayes. O’Reilley Publishing.
%                 \item Downey, A. (2014). Think stats. O’Reilley Publishing.
%                 \item McKay, D. (2003). Information theory, inference and learning algorithms. Cambridge University Press.
%                 \item Reinhart, A. (2015). Statistics done wrong. No Starch Press.
%             \end{enumerate}
%     \end{posterboxenv}

% \end{tcbposter}

% \pagebreak

% % === Page 3 =================================================================
% \begin{tcbposter}[
%         poster = {spacing=0.6em, columns=12},
%         boxes = {},
%         coverage={left=0pt, right=0pt, top=0pt, bottom=0pt},
%     ]

%     % === References ================================================================
%     \begin{posterboxenv}[skin=sectionboxskin, adjusted title=References]
%         {name=overview,column=9,below=top, span=4}
%         \printbibliography[heading=none]

%     \end{posterboxenv}

% \end{tcbposter}

% \printbibheading
% % \printbibliography[type=book,heading=subbibliography,title={Book Sources}]
% \printbibliography[nottype=book,heading=subbibliography,title={Other Sources}]

% \end{document}
